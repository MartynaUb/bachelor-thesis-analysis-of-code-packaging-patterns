\section{Įrankiai šablonų analizei ir įvertinimui}
Kompiuterinės sistemos kurioms yra aktualu klasių ir paketų skirstymo metodai, paprastai yra labai didelės.
Pilnai perprasti tokias sistemas, nustatyti jų įgyvendinimo kokybę, naudojamus šablonus kodui skirstyti,
apskaičiuoti paketų kokybes metrikas yra sudėtingas procesas.
Daryti tai rankiniu budu yra labai sudėtinga, užtrunka daug laiko bei yra paliekama daug vietos potencialioms klaidoms,
todėl yra būtina šį procesą optimizuoti, skaitmenizuoti analizės procedūras pasitelkiant,
aiškiai apibrėžtų ir programiškai efektyvių programinių įrankių pagalbą.

\subsection{Reikalavimai įrankiams}
Įrankių, leidžiančių paprasčiau atlikti sistemų analizę, atsakomybes galima suskirstyti į kelias grupes:
\begin{itemize}
    \item Bendrinė sistemos analizė - įrankis ar įrankiai turi atlikti bendrinę sistemos analize.
    Šios atsakomybių grupės įrankių išvestis nėra objektyvūs, tiksliai apibrėžtų formulių rezultatai, o papildoma, aiškiai
    atvaizduota, meta informacija apie sistema - paketų struktūrą, jų priklausomybes, figuruojančių paketų bei klasių vardai.
    Ši papildoma informacija padeda analizę atliekančiams asmeniui priimti ižvalgas apie sistemą, kaip pavyzdžiui,
    kokiam paketų skirstymo šablonui yra atimiausia sistemos strukūra,
    arba kaip lengvai sistema yra suprantama.
    \item Paketų kokybės metrikų skaičiavimas - įrankis ar įrankiai turi galėti paskaičiuoti aprašytas paketų kokybes metrikas.
    Šių įrankių išvestis - formulėmis pagrįstų skaičiavimų rezultatai apie paketų kokybę, kurie suprantamu formatu išvedami įrankio naudotojui,
    leidžiant juos palyginti su kitais rezultatais.
\end{itemize}
\subsection{Įrankių įgyvendinimas}