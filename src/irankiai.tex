\section{Įrankiai šablonų analizei ir įvertinimui}
Kompiuterinės sistemos, kurioms yra aktualu klasių ir paketų skirstymo metodai, paprastai yra labai didelės.
Pilnai perprasti tokias sistemas, nustatyti jų įgyvendinimo kokybę, naudojamus šablonus kodui skirstyti,
apskaičiuoti paketų kokybes metrikas yra sudėtingas procesas.
Daryti tai rankiniu budu užtrunka daug laiko bei yra paliekama daug vietos potencialioms klaidoms,
todėl yra būtina šį procesą optimizuoti, skaitmenizuoti analizės procedūras pasitelkiant
aiškiai apibrėžtų ir programiškai efektyvių programinių įrankių pagalbą.
Šiame darbe nagrinėjamos Java programavimo kalba parašytos sistemos.

\subsection{Reikalavimai įrankiams}
Įrankių, leidžiančių paprasčiau atlikti sistemų analizę, atsakomybes galima suskirstyti į dvi grupes:
\begin{itemize}
    \item Bendrinė sistemos analizė - įrankis ar įrankiai padeda atlikti bendrinę sistemos analizę.
    Šios atsakomybių grupės įrankių išvestis nėra objektyvūs, tiksliai apibrėžtų formulių rezultatai, o papildoma, aiškiai
    pavaizduota, meta informacija apie sistemą - paketų struktūrą, jų priklausomybes, figuruojančių paketų bei klasių vardus.
    Ši papildoma informacija nėra aiškūs teiginiai, o tik pagalba analizę atliekančiams asmeniui, leidžianti priimti ižvalgas apie sistemą,
    kaip pavyzdžiui, kokiam paketų skirstymo šablonui yra atimiausia sistemos strukūra, arba kaip lengvai sistema yra suprantama.
    \item Paketų kokybės metrikų skaičiavimas - įrankis ar įrankiai turi padeda apskaičiuoti aprašytas paketų kokybės metrikas.
    Šių įrankių išvestis - tikslūs, formulėmis pagrįstų skaičiavimų rezultatai apie paketų kokybę, kuriuos galima lyginti tarpusavyje.
\end{itemize}

\subsection{Reikalavimai bendrinės sistemos analizės įrankiui}
Įrankis bendrinei sistemos analizei atlikti turėtų suteikti galimybę naudotojui nurodyti kelią iki \textit{Java} programavimo kalba parašytos sistemos arba posistemės ir joje
atlikti jos turinio analizę bei naudotojui pateikti naudingas išvadas, sudarytas iš:
\begin{itemize}
    \item Klasių ir paketų skaičiaus
    \item Vidutinio klasių pakete skaičiaus
    \item Paketų ir klasių medį, identifikuojantį abstrakčias klases ar sąsajas
    \item Paketų priklausomybių grafiką
\end{itemize}
Gautą rezultatą išvesti suprantamu formatu, leidžiant vartotojui susidaryti išvadas apie sistemos, arba tam tikros
posistemės struktūrą, naudojamus įrankius bei kokybę.

\subsection{Reikalavimai įrankiui paketo kokybėi skaičiuoti}
Įrankis paketo kokybei skaičiuoti, turėtų suteikti galimybę naudotojui nurodyti kelią iki \textit{Java} programavimo kalba parašytos sistemos arba posistemės ir joje
apskaičiuoti kiekvieno paketo kokybės metrikas:
\begin{itemize}
    \item Klasių skaičių
    \item Aferentinių jungčių skaičių
    \item Eferentinių jungčių skaičių
    \item Nestabilumo santykį
    \item Abstrakcijos santykį
    \item Atstumo nuo pagrindinės sekos santykį
    \item Žiedinių priklausomybių skaičių
\end{itemize}
Gautą rezultatą išvesti vartotojui suprantamu formatu, kuriame matytųsi individualių paketų metrikos, bei šių metrikų vidurkis sistemoje (arba posistemėje).
Išvedimo formatas turėtų būti toks, jog skirtingų analizių rezultatai būtų lengvai palyginami su kitais.

\textbf{Abiejose vienas iš palaikomų išvesties formatų turėtų būti \textit{latex}, taip suteikiant galimybę analizės rezultatus pateikti tolesniame šio dokumento turinyje.}

\subsection{Įrankių įgyvendinimas}
Nors beveik visiems reikalavimuose minimiems funkcionalumams galima rasti jau sukurti įrankių, greit ir efektyviai pritaikyti juos skirtingoms sistemoms
(arba posistemėms) nėra patogu - kiekvieną įrankį reikėtų vykdyti atskirai, su skirtingais vykdymo procesais ir argumentais. Taip išvestų rezultatų formatai yra skirtingi.
Todėl, norint palengvinti ši procesą - suvienodinti procesų vykdymą bei gautus rezultus, visi įrankiai reikalingi analizei įgyvendinti kaip viena programinė sistema, kuri
apdoroja failus nurodytoje sistemos direktorijoje, nuskaito \textit{java} failų turinį ir sukonstruoja informaciją apie sistemos paketus bei klases.
Surinkta informacija naudojama įgyvendinti kiekvienam aprašytam įrankio funkcionalumui, ten, kur galima, naudojant jau parašytus įrankius, taip programiškai supaprastinant
skirtingų įrankių vykdymą.
