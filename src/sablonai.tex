\section{Kodo skirstymo į paketus šablonai}
Renkantis kodo skirstymo į paketus metodą, dažniausiai rekomenduojamas skirstymas pagal dalykinės srities esybes~\cite{DomainDrivenDesign}.
Tokį skirstymą galima apibrėžti kaip paketų sudarymą iš klasių ar sąsajų, kurios tarpusavyje yra konceptualiai susijusios~\cite{Functional}.
Šis skirstymo būdas dažniausiai bent jau iš dalies naudojamas net sistemose, kurios neturi aiškiai apibrėžto skirstymo būdo.
Tačiau šis metodas negali būti vienareikšmiškai pritaikomas kiekvienoje situacijoje.
Tai lemia, kad praktikoje jo nėra griežtai laikomasi - klasės būna išskaidytos remiantis papildomomis taisyklėmis,
siekiant išspręsti sistemos planavimo metu kylančias problemas.
Norint išskirti šias papildomas taisykles, šablonus, nagrinėjamos atviro kodo sistemos,
stebimi nukrypimai nuo bendresnio kodo skirstymo būdo ir identifikuojami klausimai arba problemos, kurias jais buvo bandoma išspręsti.

\subsection{Šablonų paieškos procesas}
Praktikoje naudojamų kodo skirstymo į paketus šablonų paieška buvo atliekama stebint atviro kodo sistemų struktūrą ir bandant nustatyti, kodėl pasirenkama nukrypti
nuo pagrindinio skirstymo būdo.
Nagrinėjamos atviro kodo sistemos, pasirenkant skirtingo tipo projektus, siekiant
objektyvesnės šablonų analizės skirtingose srityse.
Galimi tipai:
\begin{itemize}
    \item Taikomoji programinė įranga, teikianti paslaugas įrangos naudotojams. Pavyzdžiui,
    internetinė programėlė priminimams ir darbams užsirašyti
    \item Techninė programinė įranga, naudojama taikomosios programinės įrangos duomenų
    saugojimui, siuntimui, paieškai. Pavyzdžiui, duomenų bazės, pranešimų eilės, talpyklos
    (angl. cache)
    \item Programinės įrangos įrankiai, skirti naudoti kitose sistemose supaprastinant programinį
    kodą, naudojant jau įgyvendintas funkcijas. Pavyzdžiui, Java programavimo kalbos
    Spring karkasas internetinių programėlių kūrimui
\end{itemize}
Pastebėta, kad nukrypimai nuo skirstymo pagal dalykinės srities esybes - pagrindinio skirstymo būdo, dažnai pasikartodavo bandant išspręsti tas pačias problemas,
tačiau jų sprendimo būdai buvo skirtingi.

\subsection{Sistemos planavimo metu kylančios problemos}
Šiame skyriuje pateikiamos analizuotose sistemose rasti kodo skirstymo į paketus šablonai, kartu su jais sprendžiamomis problemomis ir pavyzdžiais,
kokiose sistemose jie yra.
Pateiktas šablonų ir problemų sąrašas nėra baigtinis, tačiau minimi šablonai pasikartoja ne vienoje sistemoje.

\subsubsection{Pagalbinių, daugkartinio naudojimo klasių skirstymas}
Pagalbinės klasės galėtų būti apibrėžiamos kaip tiesiogiai su dalykine sritimi nesusijusios, techninių funkcijų klasės,
pavyzdžiui, atsakingos už masyvų filtravimą, darbą su failais.
Pagalbinės ir daugkartinio naudojimo klasės dažniausiai negali būti patogiai grupuojamos skirstant pagal dalykinės srities esybes - pagalbines klases gali
naudoti kelios esybės, todėl neaišku, prie kurios jas reikėtų priskirti.
Vienas iš šablonų, sprendžiančių šią problemą - turėti vieną paketą, skirtą visoms pagalbinėms klasėms.
Pakete galima turėti atskiras klases kiekvienai bendrinei dalykinei sričiai, iš kurios pavadinimo programuotojas galėtų nuspresti,
kad jo ieškomas funkcionalumas bus būtent toje klasėje (pavyzdys~\ref{fig:utils}).
\begin{figure}[H]
\snugshade
\dirtree{%
    .1 {/} .
    .2 {users} .
    .3 {UserRolesGrouping} .
    .2 {\ldots} .
    .2 {payments} .
    .3 {PaymentScheduler} .
    .2 {\ldots} .
    .2 {transactions} .
    .3 {TransactionBatching} .
    .2 {utils} .
    .3 {Arrays} .
    .3 {Maps} .
    .3 {SqlQueries} .
    .3 {Users} .
}
\endsnugshade
\caption{Sistemos pavyzdys, kur visas bendrinio panaudojimo kodas guli \textit{utils} pakete, pirmame sistemos paketų lygyje}
\label{fig:utils}
\end{figure}

Jei pagalbinių klasių dydis labai išauga, galima vietoje vienos atskiros klasės vienai bendrinei sričiai sukurti vieną paketą,
ir jame turėti kelias pagalbines klases, susijusias su ta dalykine sritimi (pavyzdys~\ref{fig:common}).
\begin{figure}[H]
\snugshade
\dirtree{%
    .1 {/common} .
    .2 {arrays} .
    .3 {ArrayFilters} .
    .3 {ArrayComparators} .
    .2 {\ldots} .
    .2 {maps} .
    .3 {MapTransformations} .
    .3 {MapJoining} .
    .2 {json} .
    .3 {JsonParser} .
    .2 {\ldots} .
    .2 {database} .
    .3 {DatabaseConnection} .
    .3 {DatabaseQueries} .
}
\endsnugshade
\caption{Sistemos pavyzdys, kur bendrinio panaudojimo kodas guli \textit{common} pakete, po dalykinės srities subpaketais, taip sumažinant klasių dydį }
\label{fig:common}
\end{figure}
Tokį pagalbinių klasių skirstymo metodą naudoja keletas repozitorijų - pavyzdžiui, užrašinės aplikacijos \textit{Omni-Notes}\footnote{\url{https://github.com/federicoiosue/Omni-Notes/tree/develop/omniNotes/src/main/java/it/feio/android/omninotes/utils}} bei
\textit{Fire Sticker}\footnote{\url{https://github.com/hackjutsu/Fire_Sticker/tree/master/app/src/main/java/com/gogocosmo/cosmoqiu/fire_sticker/Utils}}.
\textit{Omni-Notes} atveju, \textit{utils} pakete esančios klasės papildomai skirstomos pagal atitinkamą dalykinę sritį.
Šio šablono aprašymas (\ref{table:atskiras})
\begin{figure}[H]
\begin{center}
    \begin{tabular}{|p{5cm}|p{10cm}|}
        \hline
        Šablonas & Atskiras pagalbinių klasių paketas \\ [0.5ex]
        \hline\hline
        Sprendžiama problema & Pagalbinių klasių skirstymas\\
        \hline
        Siekiamybė & Mažesnis kodo dubliavimas, lengvai randamos pagalbinės klasės \\
        \hline
        Siūlomas sprendimas & Sukurti atskirą pagalbinių klasių paketą dalykinės srities esybių lygmenyje \\
        \hline
        Galimos variacijos & Gali būti papildomai sukuriami subpaketai pagal pagalbinių klasių teikiamą funkcionalumą \\
        \hline
    \end{tabular}
\end{center}
\caption{\textit{Atskiro pagalbinių klasių paketo} šablonas}
\label{table:atskiras}
\end{figure}

Šiai problemai spręsti galimas ir kitoks būdas - turėti keletą pagalbinių klasių paketų, patalpintų po atitinkamu bendresniu dalykinės srities
 paketu.
Tokį skirstymo į paketus šabloną naudoja transakcijoms skirta programinė įranga \textit{Seata}\footnote{\url{https://github.com/apache/incubator-seata/tree/2.x/saga/seata-saga-engine/src/main/java/org/apache/seata/saga/engine/utils}}.
Čia pagalbinės klasės yra \textit{utils} pakete žemesniame sistemos lygmenyje (pavyzdys~\ref{fig:domainHelpers}), pavyzdžiui, po \textit{engine} paketu.
\begin{figure}[H]
    \snugshade
    \dirtree{%
        .1 {/engine} .
        .2 {strategy} .
        .2 {sequence} .
        .2 {\ldots} .
        .2 {utils} .
        .3 {ExceptionUtils} .
        .3 {\ldots} .
    }
    \endsnugshade
    \caption{Sistemos pavyzdys, kur bendrinio panaudojimo kodas guli \textit{utils} pakete esančiame po dalykinės srities subpaketais}
    \label{fig:domainHelpers}
\end{figure}
Šio šablono aprašymas (\ref{table:domainHelper})
\begin{figure}[H]
\begin{center}
    \begin{tabular}{|p{5cm}|p{10cm}|}
        \hline
        Šablonas & Pagalbinių klasių priskyrimas dalykinės srities paketui \\ [0.5ex]
        \hline\hline
        Sprendžiama problema & Pagalbinių klasių skirstymas\\
        \hline
        Siekiamybė & Kiekvienai esybei priskirtos pagalbinės klasės\\
        \hline
        Siūlomas sprendimas & Klases laikyti \textit{util} paketuose, žemesniame lygmenyje nei dalykinės srities esybės \\
        \hline
    \end{tabular}
\end{center}
\caption{\textit{Pagalbinių klasių priskyrimo dalykinės srities paketui} šablonas}
\label{table:domainHelper}
\end{figure}

\subsubsection{Didelis klasių skaičius pakete}
Sistemai plečiantis, neišvengiamai didėja ir klasių skaičius. Net pasirinkus klasių skirstymo į paketus metodą
ir palaikant šią tvarką, klasių skaičius pakete gali išaugti. Tokį pavyzdį galima matyti paieškos įrankio \textit{Elasticsearch}\footnote{\url{https://github.com/elastic/elasticsearch/tree/main/server/src/main/java/org/elasticsearch/transport}}
\textit{transport} pakete. Nors klasės yra sugrupuotos pagal dalykinės srities esybę, bendras jų skaičius pakete siekia beveik devyniasdešimt.
Tokiame pakete naviguoti sunku, jis taip pat gali turėti daug priklausomybių.
Jei tokia tendencija yra būdinga visai sistemai, sistema tampa mažiau lanksti pokyčiams, kadangi net ir paprastas pakeitimas
daro įtaką reišmingai sistemos daliai, pokyčiai yra labiau linkę keisti bendrą sistemos architektūrą.
Taip pat naujos sistemos versijos išleidimo \angl{release} procesas tampa sudetingesnis, kadangi yra paveikiama daugiau klasių.
Dažnai matoma, kad susidarius šiai situacijai, nukrypstama nuo skirstymo pagal dalykinės srities esybes, pradedamos grupuoti
techninio funkcionalumo klasės. Tokį skirstymą galima matyti \textit{DBeaver} duomenų bazės įrankyje \footnote{\url{https://github.com/dbeaver/dbeaver/tree/devel/plugins/org.jkiss.dbeaver.ext.mssql/src/org/jkiss/dbeaver/ext/mssql}} -
aukštesniame lygmenyje grupuojant pagal dalykinės srities esybes, pavyzdžiui, \textit{MsSql}, vėliau pereinama prie skirstymo
pagal techninį funkcionalumą pakete \textit{model}.
Šio šablono aprašymas (\ref{table:technical})
\begin{figure}[H]
\begin{center}
    \begin{tabular}{|p{5cm}|p{10cm}|}
        \hline
        Šablonas & Žemesnio lygio paketų sudarymas grupuojant pagal techninį funkcionalumą \\ [0.5ex]
        \hline\hline
        Sprendžiama problema & Didelis klasių skaičius pakete\\
        \hline
        Siekiamybė & Išskirti dalykinės srities esybės techninius funkcionalumus \\
        \hline
        Siūlomas sprendimas & Dalykinės srities esybių klases skirstyti pagal techninį funkcionalumą \\
        \hline
    \end{tabular}
\end{center}
\caption{\textit{Žemesnio lygio paketų sudarymas grupuojant pagal techninį funkcionalumą} šablonas}
\label{table:technical}
\end{figure}

Kitą skirstymo būdą galima stebėti \textit{IntelliJ Source Wizard} įskiepio kode \footnote{\url{https://github.com/wix-incubator/source-wizard/tree/master/src/main/kotlin/plugin}}.
Čia \textit{plugin} paketo esybė yra papildomai išskirstyta pagal \textit{IntelliJ} sistemos dalykinės srities esybes.
Vadovaujantis šiuo šablonu, klasės skirstomos pagal smulkesnį jų teikiamą funkcionalumą - \textit{actions}, \textit{extensions}, \textit{ui},
šiame kontekste yra ne techniniai sluoksniai, o \textit{Intellij} įskiepio dalykinės srities dalis (pavyzdys~\ref{fig:modules}).
\begin{figure}[H]
    \snugshade
    \dirtree{%
        .1 {/plugin} .
        .2 {actions} .
        .2 {extensions} .
        .2 {ui} .
    }
    \endsnugshade
    \caption{\textit{Source Wizard} sistemos dalis, kurioje matomi smulkiau išskirstyti moduliai}
    \label{fig:modules}
\end{figure}

Šio šablono variacija - smulkūs moduliai su atskirose klasėse aprašytais kontraktais.
Funkcionalumas kitiems paketams galėtų būti pasiekiamas per vieną minimalią sąsają \angl{interface},
kuri atskleidžia tik konceptus (metodus arba duomenų tipus), kurie yra glaudžiai susiję su komponento teikiama paslauga, bei
klase, grąžinančią minėtos sąsajos įgyvendinimą.
Tai gali būti paprasta klasė, su statine funkcija, kurios rezultatas yra ši sąsaja, arba, esant keliomis sąsajos implementacijomis,
\textit{Static factory} dizaino šablonas, kuris nusprendžia, kurį įgyvendinimą reikėtų grąžinti pagal paduotus argumentus.
Visos kitos klasės naudoja \textit{package} pasiekiamumo modifikatorių, taip kompiliatoriaus pagalba užtikrinant,
kad jos nebus pasiektos iš išorės.
Toks principas yra sutinkamas \textit{Typescript} programavimo kalboje, kur modulis (šios kalbos paketo atitikmuo) gali turėti
\textit{index.ts} failą, veikiantį kaip sąsaja, apibrėžianti, kokios modulio klasės bei funkcijos gali būti pasiekiamos už modulio ribų.
Toki šabloną naudoja turinio valdymo sistema \textit{Keystone}\footnote{\url{https://github.com/keystonejs/keystone/tree/main/packages/core/src/fields/types}},
kur kiekviena esybė turi savo modulį, atliekantį vieną funkciją, kurios kontraktas aprašytas \textit{index.ts} faile (pavyzdys~\ref{fig:kontraktai}).

\begin{figure}[H]
    \snugshade
    \dirtree{%
        .1 {/types} .
        .2 {image} .
        .3 {views} .
        .3 {utils} .
        .3 {index} .
        .2 {text} .
        .3 {views} .
        .3 {index} .
        .2 {...} .
        .2 {checkbox} .
        .3 {views} .
        .3 {index} .
    }
    \endsnugshade
    \caption{\textit{Keystone} sistemos dalis, kurioje matomi smulkūs moduliai su \textit{index.ts} failuose aprašytais kontraktais}
    \label{fig:kontraktai}
\end{figure}

Šio šablono aprašymas (\ref{table:module})
\begin{figure}[H]
\begin{center}
    \begin{tabular}{|p{5cm}|p{10cm}|}
        \hline
        Šablonas & Skirstymas pagal smulkų funkcionalumą \\ [0.5ex]
        \hline\hline
        Sprendžiama problema & Didelis klasių skaičius pakete\\
        \hline
        Siekiamybė & Maži paketai, turintys aiškiai apibrėžtą funkcionalumą\\
        \hline
        Siūlomas sprendimas & Klases suskirstyti pagal smulkesnį jų teikiamą funkcionalumą \\
        \hline
        Galimos variacijos & Smulkūs moduliai su atskirose klasėse aprašytais kontraktais \\
        \hline
    \end{tabular}
\end{center}
\caption{\textit{Skirstymo pagal smulkų funkcionalumą} šablonas}
\label{table:module}
\end{figure}

\subsubsection{Skirtingų sąsajų implementacijų skirstymas}
Sistemai plečiantis galima susidurti su problema, kad išauga sąsajos įgyvendinimų skaičius.
Ši problema aktuali skirstant klases pagal dalykinės srities esybes - laikant sąsajų įgyvendinimus
skirtinguose esybių paketuose (kai sąsaja nėra susijusi su viena esybe, pavyzdžiui, ją įgyvendina kelios esybės),
navigacija paketuose pasidaro sudėtinga, neaišku, kas kurią klasę įgyvendina.
Galimas šios problemos sprendimas - sukurti sąsajos implementacijas tame pačiame pakete kaip ir pati sąsaja.
Toks skirstymo į paketus šablonas yra sutinkamas \textit{Java} standartinėje bibliotekoje\footnote{\url{https://github.com/AdoptOpenJDK/openjdk-jdk11/tree/master/src/java.base/share/classes/java/util}},
kur visos standartinės sąsajos bei jų implementacijos yra tame pačiame pakete module (pavyzdys~\ref{fig:implJava}).
\begin{figure}[H]
    \snugshade
    \dirtree{%
        .1 {/util} .
        .2 {Map} .
        .2 {HashMap} .
        .2 {LinkedHashMap} .
        .2 {WeakHashMap} .
        .2 {TreeMap} .
        .2 {...} .
        .2 {List} .
        .2 {LinkedList} .
        .2 {ArrayList} .
    }
    \endsnugshade
    \caption{\textit{Java} standartinės bibliotekos pavyzdys, kuriame sąsajų įgyvendinimai guli tame pačiame pakete, kaip ir sąsajos}
    \label{fig:implJava}
\end{figure}

Šio šablono aprašymas (\ref{table:implGr})
\begin{figure}[H]
\begin{center}
\begin{tabular}{|p{5cm}|p{10cm}|}
    \hline
    Šablonas & Sąsajų ir implementacijų grupavimas \\ [0.5ex]
    \hline\hline
    Sprendžiama problema & Daug skirtingų sąsajų implementacijų\\
    \hline
    Siekiamybė & Susieti sąsają ir jos implementacijas lengvesnei jų paieškai\\
    \hline
    Siūlomas sprendimas & Sąsajų įgyvendinimus laikyti tame pačiame pakete, kaip ir sąsajas \\
    \hline
\end{tabular}
\end{center}
\caption{\textit{Sąsajų ir implementacijų grupavimo} šablonas}
\label{table:implGr}
\end{figure}
Kitas galimas sprendimo būdas - sąsaja, kuri turi daug skirtingų įgyvendinimų, turėtų turėti specifiškai jai sukurtą paketą, o kiekvienam paketo
įgyvendinimui sukuriama po subpaketą, kuriame yra tiek klasę įgyvendinanti sąsają, tiek kitos, įgyvendinimui reikalingos klasės.
Tokia paketų struktūra buvo sutikta tokiose sistemose kaip \textit{HikariCP}\footnote{\url{https://github.com/brettwooldridge/HikariCP/tree/dev/src/main/java/com/zaxxer/hikari}},
skirtoje efektyviam \textit{Java} programos prisijungimui prie duomenų bazių, bei
\textit{Leaf}\footnote{\url{https://github.com/Meituan-Dianping/Leaf/tree/master/leaf-core/src/main/java/com/sankuai/inf/leaf}} (pavyzdys~\ref{fig:leaf}), naudojamoje unikalaus identifikatoriaus generavimui.

\begin{figure}[H]
    \snugshade
    \dirtree{%
        .1 {/} .
        .2 {segment} .
        .3 {dao} .
        .3 {model} .
        .3 {SegmentIdGenImpl} .
        .2 {snowflake} .
        .3 {SnowflakeIdGenImpl} .
        .3 {SnowflakeZookeeperHolder} .
        .2 {IdGen} .
    }
    \endsnugshade
    \caption{\textit{Leaf} sistema, kur kiekvienas sąsajos įgyvendinimas yra aprašytas subpakete}
    \label{fig:leaf}
\end{figure}
Šio šablono aprašymas (\ref{table:implSub})
\begin{figure}[H]
\begin{center}
    \begin{tabular}{|p{5cm}|p{10cm}|}
        \hline
        Šablonas & Įgyvendinimų atskyrimas \\ [0.5ex]
        \hline\hline
        Sprendžiama problema & Daug skirtingų sąsajų implementacijų\\
        \hline
        Siekiamybė & Skirtingų sąsajų bei implementacijų atskyrimas\\
        \hline
        Siūlomas sprendimas & Kiekvienam paketo įgyvendinimui sukurti subpaketą, kuriame yra tiek klasę įgyvendinanti sąsaja, tiek kitos, įgyvendinimui reikalingos klasės \\
        \hline
    \end{tabular}
\end{center}
\caption{\textit{Įgyvendinimų atskyrimo} šablonas}
\label{table:implSub}
\end{figure}
\subsubsection{Esybių pokyčių ir versijavimo valdymas}
Sistemai egzistuojant ilgesnį laiką, jos pokyčiai pasidaro neišvengiami.
Smulkūs pokyčiai nebūtinai paveikia bendrą sistemos struktūrą, tačiau
didesniems pokyčiams kartais reikia sukurti naujas klasių ar funkcionalumų versijas. Jei reikia palaikyti atgalinį
suderinamumą (angl. backward compatibility), sistemoje gali atsirasti kelios tų pačių esybių versijos.
Tokiu atveju sistemoje egzistuojančios kelios tų pačių esybių versijos gali pridėti painumo.
Ši problema gali būti sprendžiama sukuriant atskirus paketus skirtingoms
versijoms (v1, v2, v3, \ldots) bei iškeliant bendrą, abiejų esybių naudojamą kodą į paketus taip, kad jas galėtų pasiekti abi versijos.
\begin{figure}[H]
    \snugshade
    \dirtree{%
        .1 {/regex} .
        .2 {common} .
        .2 {pending} .
        .2 {v3} .
        .2 {v4} .
        .2 {...} .
    }
    \endsnugshade
    \caption{\textit{Mongo} duomenų bazės pavyzdys, kuriame skirtingos versijos patalpintos atskiruose paketuose}
    \label{fig:mongo}
\end{figure}
Toks skirstymo būdas matomas keliose repozitorijose - \textit{Mongo}\footnote{\url{https://github.com/mongodb/mongo/tree/master/src/third_party/boost/boost/regex}} duomenų bazėje (pavyzdys~\ref{fig:mongo}),
API skirtame kelionių valdymui \textit{travels-java-api}\footnote{\url{https://github.com/mariazevedo88/travels-java-api/tree/master/src/main/java/io/github/mariazevedo88/travelsjavaapi/controller}}
 bei duomenų saugojimo įrankyje \textit{nocodb}\footnote{\url{https://github.com/nocodb/nocodb/tree/c7cc1f92fd77f8b5daefceb7148aab4a69cb9b4e/packages/nocodb/src/meta/migrations}}.
Šiose repozitorijose v1, v2 (\textit{Mongo} atveju - v4, v5) pavadinti paketai saugo skirtingas esybių versijas.
Šio šablono aprašymas (\ref{table:versions})
\begin{figure}[H]
\begin{center}
    \begin{tabular}{|p{5cm}|p{10cm}|}
        \hline
        Šablonas & Atskirų versijų skirstymas \\ [0.5ex]
        \hline\hline
        Identifikatorius & \textit{versijos-1} \\
        \hline
        Sprendžiama problema & Esybių pokyčių ir versijavimo valdymas\\
        \hline
        Siekiamybė & Palaikyti skirtingas esybių versijas \\
        \hline
        Siūlomas sprendimas & Sukurti atskirus paketus skirtingoms versijoms bei iškelti bendrą, abiejų esybių naudojamą kodą į paketus \\
        \hline
    \end{tabular}
\end{center}
\caption{\textit{Atskirų versijų skirstymo} šablonas}
\label{table:versions}
\end{figure}


