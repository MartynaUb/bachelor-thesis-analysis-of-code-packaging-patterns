\section{Esamų sistemų pertvarkymas}
Apibrėžus šablonus paketų skirstymui galima pradėti vertinti jų įtaką sistemai, juos pritaikant esamoms sistemoms.
Šio skyriaus tikslas - pasirinkti ir išnagrinėti kelias sistemas, pasitelkus paketų kokybės matus, bei bendrą sistemos struktūros analizę,
taip identifikuojant programiniam kodui būdingas problemas, rastas problemas išspręsti pritaikant aprašytus šablonus.
Atlikus pertvarkymus sistemose, dar kartą paskaičiuoti paketų kokybės matus,
taip gaunant įrodymus, ar gauti šablonai yra efektyvūs ir iš tiesų sprendžia
jiems priskirtas problemas.


\subsection{Sistemų pasirinkimas}
Sistemų pertvarkymui pasirinktos atviro kodo sistemos, kurių kodas yra viešai prietinamas \textit{github} platformoje.
Pasirinktos sistemos yra vidutinio dydžio, todėl nėra labai sudėtinga jas suprasti ir pertvarkyti, bet taip pat jos nėra
tokios paprastos, kad neturėtų sistemos dizaino problemų.
Pasirinktos sistemos yra skirtingo tipo projektai, taip užtikrinant didesnę problemų ivairovę ir objektyvesnius įvertinimus.
Per visą pasirinktų sistemų imtį yra sutinkamos visos aprašytos problemos, taip įvertinant visus aprašytus šablonus.

\subsection{\textit{Leaf} sistema}
\textbf{Leaf\footnote{\url{https://github.com/Meituan-Dianping/Leaf/tree/master}}} sistema, per \textit{http}
protokolą teikianti aplikacijų programavimo sąsają unikalaus identifikatoriaus generavimui.
Ši sistema užtikrina identifikatoriaus unikalų pasiskirstymą tarp skirtingų paskirstytų sistemų \angl{distributed systems},
servisais orientuotoje \angl{service-oriented} architektūroje.
Tai techninės programinės įrangos tipas, kuris naudojamas kitų taikomosios programinės įrangos sistemų.
Sistema Leaf susideda iš dviejų modulių \textit{server} ir \textit{core}, šiame darbe dėmesys bus skirtas tik \textit{core} moduliui,
kadangi \textit{server} modulis yra labai mažas
ir paketų struktūra jame neatlieka esminio vaidmens.
Prieš visus pakeitimus \textit{Leaf} sistemos, \textit{core} modulio paketų struktūra atrodo taip:
\begin{figure}[H]
    \centering
    \includegraphics[scale=0.5]{img/leaf_packages_orig}
    \caption{\textit{Leaf} sistemos \textit{core} modulio struktūra}
    \label{img:leaf_packages_orig}
\end{figure}

\dirtree{%
.1 {/}.
.2 {common}.
.2 {segment}.
.3 {dao}.
.4 {impl}.
.3 {model}.
.2 {snowflake}.
.3 {exception}.
}

\begin{center}
    \begin{tabular}{|c|c|c|c|c|c|c|}
        \hline
        Paketo vardas & \textit{N} & \textit{A} & \textit{E} & \textit{S} & \textit{A} & \textit{D} \\ [0.5ex]
        \hline\hline
        leaf.segment.dao.impl & 1 & 0 & 2 & 1.0 & 0.0 & 0.0 \\
        \hline
        leaf.segment & 1 & 0 & 4 & 1.0 & 0.0 & 0.0 \\
        \hline
        leaf.snowflake.exception & 3 & 1 & 0 & 0.0 & 0.0 & 1.0 \\
        \hline
        leaf.segment.model & 3 & 3 & 0 & 0.0 & 0.0 & 1.0 \\
        \hline
        leaf.segment.dao & 2 & 2 & 1 & 0.333 & 1.0 & 0.333 \\
        \hline
        leaf & 1 & 3 & 1 & 0.25 & 1.0 & 0.25 \\
        \hline
        leaf.common & 6 & 3 & 1 & 0.25 & 0.0 & 0.75 \\
        \hline
        leaf.snowflake & 2 & 0 & 3 & 1.0 & 0.0 & 0.0 \\
        \hline
    \end{tabular}
    \begin{tabular}{|c|c|c|c|c|c|}
        \hline
        $\bar{N}$ & $\bar{A}$ & $\bar{E}$ & $\bar{S}$ & $\bar{A}$ & $\bar{D}$ \\ [0.5ex]
        \hline\hline
        2 & 2 & 2 & 0.479 & 0.25 & 0.417 \\
        \hline
    \end{tabular}
\end{center}


\subsection{\textit{HikariCP} sistema}

\begin{figure}[H]
    \centering
    \includegraphics[scale=0.5]{img/hikari_packages_orig}
    \caption{\textit{Hikari} sistemos struktūra}
    \label{img:hikari_packages_orig}
\end{figure}


\dirtree{%
    .1 {/} .
    .2 {hikari}.
    .3 {hibernate}.
    .3 {metrics}.
    .4 {dropwizard}.
    .4 {micrometer}.
    .4 {prometheus}.
    .3 {pool}.
    .3 {util}.
}


\begin{center}
    \begin{tabular}{|c|c|c|c|c|c|c|}
        \hline
        Paketo vardas & \textit{N} & \textit{Af} & \textit{Ef} & \textit{S} & \textit{A} & \textit{D} \\ [0.5ex]
        \hline\hline
        com.zaxxer.hikari.metrics & 4 & 5 & 0 & 0.0 & 0.75 & 0.25 \\
        \hline
        com.zaxxer.hikari.metrics.prometheus & 5 & 0 & 1 & 1.0 & 0.0 & 0.0 \\
        \hline
        com.zaxxer.hikari & 6 & 4 & 3 & 0.429 & 0.5 & 0.071 \\
        \hline
        com.zaxxer.hikari.metrics.micrometer & 2 & 1 & 1 & 0.5 & 0.0 & 0.5 \\
        \hline
        com.zaxxer.hikari.metrics.dropwizard & 3 & 1 & 3 & 0.75 & 0.0 & 0.25 \\
        \hline
        com.zaxxer.hikari.pool & 12 & 3 & 5 & 0.625 & 0.583 & 0.208 \\
        \hline
        com.zaxxer.hikari.hibernate & 2 & 0 & 1 & 1.0 & 0.0 & 0.0 \\
        \hline
        com.zaxxer.hikari.util & 9 & 2 & 2 & 0.5 & 0.111 & 0.389 \\
        \hline
    \end{tabular}
    \begin{tabular}{|c|c|c|c|c|c|}
        \hline
        $\bar{N}$ & $\bar{Af}$ & $\bar{Ef}$ & $\bar{S}$ & $\bar{A}$ & $\bar{D}$ \\ [0.5ex]
        \hline\hline
        5 & 2 & 2 & 0.601 & 0.243 & 0.209 \\
        \hline
    \end{tabular}
\end{center}

\subsection{\textit{chronicle map} sistema}
\begin{center}
    \begin{tabular}{|c|c|c|c|c|c|c|}
        \hline
        Paketo vardas & \textit{N} & \textit{A} & \textit{E} & \textit{S} & \textit{A} & \textit{D} \\ [0.5ex]
        \hline\hline
        net.openhft.chronicle.map.impl.stage.data.bytes & 2 & 3 & 3 & 0.5 & 0.0 & 0.5 \\
        \hline
        net.openhft.chronicle.map & 32 & 12 & 4 & 0.25 & 0.531 & 0.219 \\
        \hline
        net.openhft.chronicle.map.locks & 2 & 0 & 1 & 1.0 & 0.5 & 0.5 \\
        \hline
        net.openhft.chronicle.map.impl.ret & 2 & 3 & 1 & 0.25 & 1.0 & 0.25 \\
        \hline
        net.openhft.chronicle.map.impl.stage.data & 2 & 5 & 2 & 0.286 & 0.0 & 0.714 \\
        \hline
        net.openhft.chronicle.map.impl.stage.entry & 2 & 5 & 5 & 0.5 & 1.0 & 0.5 \\
        \hline
        net.openhft.chronicle.map.impl & 16 & 10 & 13 & 0.565 & 0.313 & 0.122 \\
        \hline
        net.openhft.chronicle.map.replication & 3 & 5 & 1 & 0.167 & 1.0 & 0.167 \\
        \hline
        net.openhft.chronicle.map.impl.stage.query & 9 & 2 & 7 & 0.778 & 0.667 & 0.445 \\
        \hline
        net.openhft.chronicle.map.impl.stage.map & 7 & 4 & 5 & 0.556 & 0.857 & 0.413 \\
        \hline
        net.openhft.chronicle.map.internal & 3 & 1 & 0 & 0.0 & 0.0 & 1.0 \\
        \hline
        net.openhft.chronicle.map.impl.stage.ret & 2 & 2 & 1 & 0.333 & 1.0 & 0.333 \\
        \hline
        net.openhft.chronicle.map.channel & 3 & 1 & 2 & 0.667 & 0.667 & 0.334 \\
        \hline
        net.openhft.chronicle.map.impl.stage.data.instance & 1 & 3 & 1 & 0.25 & 0.0 & 0.75 \\
        \hline
        net.openhft.chronicle.map.impl.stage.iter & 6 & 1 & 7 & 0.875 & 0.333 & 0.208 \\
        \hline
        net.openhft.chronicle.map.impl.stage.replication & 2 & 5 & 3 & 0.375 & 0.5 & 0.125 \\
        \hline
        net.openhft.chronicle.map.channel.internal & 1 & 1 & 2 & 0.667 & 0.0 & 0.333 \\
        \hline
        net.openhft.chronicle.map.impl.stage.input & 1 & 1 & 6 & 0.857 & 1.0 & 0.857 \\
        \hline
    \end{tabular}
    \begin{tabular}{|c|c|c|c|c|c|}
        \hline
        $\bar{N}$ & $\bar{A}$ & $\bar{E}$ & $\bar{S}$ & $\bar{A}$ & $\bar{D}$ \\ [0.5ex]
        \hline\hline
        5 & 4 & 4 & 0.493 & 0.52 & 0.432 \\
        \hline
    \end{tabular}
\end{center}


\begin{figure}[H]
    \centering
    \includegraphics[scale=0.5]{img/open_hft_packages_orig}
    \caption{\textit{Chronicle-map} sistemos struktūra}
    \label{img:open_hft_packages_orig}
\end{figure}

\begin{center}
    \begin{tabular}{|c|c|c|c|c|c|c|c|}
        \hline
        Paketo vardas & \textit{N} & \textit{Af} & \textit{Ef} & \textit{S} & \textit{A} & \textit{D} & \textit{C} \\ [0.5ex]
        \hline\hline
        net.openhft.chronicle.hash.impl.stage.iter & 6 & 0 & 5 & 1.0 & 0.5 & 0.5 & 0 \\
        \hline
        net.openhft.chronicle.hash.impl.util.jna & 1 & 1 & 0 & 0.0 & 0.0 & 1.0 & 0 \\
        \hline
        net.openhft.chronicle.hash.impl.stage.query & 6 & 2 & 5 & 0.714 & 0.667 & 0.381 & 2 \\
        \hline
        net.openhft.chronicle.hash.internal & 1 & 0 & 0 & 0.0 & 0.0 & 1.0 & 0 \\
        \hline
        net.openhft.chronicle.hash.locks & 4 & 3 & 1 & 0.25 & 0.5 & 0.25 & 1 \\
        \hline
        net.openhft.chronicle.hash.impl.stage.replication & 1 & 0 & 1 & 1.0 & 1.0 & 1.0 & 0 \\
        \hline
        net.openhft.chronicle.hash.serialization & 11 & 5 & 2 & 0.286 & 0.636 & 0.078 & 2 \\
        \hline
        net.openhft.chronicle.hash.impl.stage.entry & 16 & 4 & 6 & 0.6 & 0.5 & 0.1 & 3 \\
        \hline
        net.openhft.chronicle.hash.replication & 4 & 2 & 1 & 0.333 & 0.5 & 0.167 & 1 \\
        \hline
        net.openhft.chronicle.hash.impl.util & 8 & 1 & 1 & 0.5 & 0.125 & 0.375 & 0 \\
        \hline
        net.openhft.chronicle.hash.impl.stage.hash & 7 & 5 & 4 & 0.444 & 0.571 & 0.015 & 2 \\
        \hline
        net.openhft.chronicle.hash & 21 & 11 & 4 & 0.267 & 0.762 & 0.029 & 3\\
        \hline
        net.openhft.chronicle.hash.impl.stage.data.bytes & 2 & 1 & 4 & 0.8 & 0.0 & 0.2 & 1 \\
        \hline
        net.openhft.chronicle.hash.serialization.impl & 48 & 3 & 2 & 0.4 & 0.083 & 0.517 & 1 \\
        \hline
        net.openhft.chronicle.hash.impl.util.math & 5 & 0 & 0 & 0.0 & 0.2 & 0.8 & 0 \\
        \hline
        net.openhft.chronicle.hash.impl & 19 & 5 & 7 & 0.583 & 0.316 & 0.101 & 2 \\
        \hline
    \end{tabular}
    \begin{tabular}{|c|c|c|c|c|c|c|}
        \hline
        $\bar{N}$ & $\bar{A}$ & $\bar{E}$ & $\bar{S}$ & $\bar{A}$ & $\bar{D}$ & $sum(C)$ \\ [0.5ex]
        \hline\hline
        10 & 3 & 3 & 0.449 & 0.397 & 0.407 & 9 \\
        \hline
    \end{tabular}
\end{center}
Matų lentelėje matomos kelios sistemos problemos -
\begin{itemize}
    \item \textbf{Problema numeris 1} - nuo paketo \textit{net.openhft.chronicle.hash} priklauso 11 kitų paketų.
    Nors pagal kokybės matus paketas yra gan stabilus - didelės abstrakcijos deka (\textit{A}=0.762), jo nuokrypis nuo pagrindinės sekos nėra didelis (\textit{D}=0.029) ir
    tokio paketo pakeitimai neturės labai didelės įtakos nuo jo priklausantiems paketams, vis dėlto dėl didelio priklauomybių skaičiaus, programuotojui yra gan sunku suprasti funkcijas,
    kuriose figuruoja šis paketas, ir kokio domeno tipo funkcionalumas pakeisti reikėtų dirbti su šiuo paketu.
    \item \textbf{Problema numeris 2} - Sistema turi labai daug ciklinių prilausomynbių kur du arba daugiau paketų priklauso vienas nuo kito.
    Tai aiškiai matosi sistemos paketų diagramoje \ref{img:open_hft_packages_orig}, ciklinėse priklausomybėse išviso dalyvauja 9 paketai ir kartu jie sudaro 9 ciklinių priklausomybių sekas,
    kurios turėtų būti pašalintos.
    \item \textbf{Problema numeris 3} -
\end{itemize}
Nors ši tai nesimato kokybės matuose, bet atlikus bendra sistemos kodo analizę matoma \textbf{problema 3} - sistemos paketuose yra daug sąsajų su skirtingais jų įgyvendinimais, todėl yra sunku
rasti visus specifinės sąsajos įgyvendinimus, esama paketų struktūra tam negelbėja.


Norint išspresti problemą numeris 1, reikėtų išskaidyti