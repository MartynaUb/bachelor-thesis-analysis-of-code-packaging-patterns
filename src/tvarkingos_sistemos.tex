\section{Kompiuterinės sistemos vertinimas}

\subsection{Teisingai įgyvendinta kompiuterinė sistema}
Norint išsiaiškinti, kokią įtaką sistemos kokybei daro skirtingos paketų skirstymo metodologijos ir kaip ojektyviai tą įtaką pamatuoti, pirmiausia
reikėtų apsibrėžti kokiais požymiais pasižymi teisingai įgyvendinta kompiuterinė sistema.
Martin Kleppmann savo knygoje \textit{Designing Data-Intensive Applications: The Big Ideas Behind Reliable, Scalable, and Maintainable Systems} išskyria šiuos pagrindinius kriterijus:
\begin{itemize}
    \item Patikimumas, reiškiantis, kad net ir klaidų (įrangos, programinių ar žmogiškųjų) atveju,
    sistema veikia stabiliai ir patikimai, paslepiant tam tikras klaidas nuo vartotojo\cite{DataIntensiveApplications}.
    \item Prižiūrimumas, reiškiantis jog skirtingų abstrakcijų pagalba sumažintas sistemos kompleksiškumas.
    Dėl to nesunku keisti esamą sistemos funkcionalumą bei pritaikyti naujiems verslo naudojimo atvejams.
    Tai supaprastina darbą inžinierių ir operacijų komandoms dirbančioms su šia sistema, taip pat leidžia prie sistemos prisidėti naujiems žmonėms, o ne
    tik jos ekspertams.
    Tai ypač aktualu atviro kodo sistemoms\cite{DataIntensiveApplications}.
    \item Plečiamumas, reiškiantis jog sistema turi strategijas, kaip išlaikyti gerą našumą užklausų
    srautui didėjant ir sistemai augant, tai atliekant su pagrįstais kompiuteriniais resursais ir
    priežiūros kaina\cite{DataIntensiveApplications}.
\end{itemize}
Yra daug skirtingų elementų, sudarančių sistemą, kuri tenkintų aukščiau paminėtus kriterijus,
pavyzdžiui, pasirinktos technologijos, aukšto lygio architektūra, dokumentacija, sistemos testavimo
procesai, jų kiekis ir pan.
Vienas iš svarbių elementų, prisidedančių prie gerai įgyvendintos sistemos dizaino yra programinio kodo dizainas, jo skaitomumas, patikimumas.
Konvencijos, kaip vadinti kodo paketus, kokias klasės jiems priskirti ir kokios paketų hierarchijos laikytis sudaro svarbią programinio kodo dizaino dalį,
Todėl programinės įrangos kurimo metu, laikas skirtas rasti sistemai tinkamą paketų skirstymo šabloną ir to šablono laikymasis atsiperka, padarant
programinį kodą geriau suprantamu, taip prisidėdant prie bendro sistemos dizaino patikimumo ir lengvesnio palaikomumo.

Straipsnyje \textit{Investigating The Effect of Software Packaging on Modular Structure Stability}, jo autoriai akcentuoja, kad
gerai įgyvendintos, objektiškai orientuotos sistemos turėtų vystytis be didelių pakeitimu jų architektūroje.
To siekiama todėl, nes architektūriniai pakeitimai paveikią didelę sistemos dalį ir todėl
jų įgyvendinimo ir priežiūros kaštai yra ženkliai didesni\cite{ModularStability}.
Paketų struktūra, kuri užtikrina atsieta (\angl{decoupled}) komunikavima tarp paketų, enkapsuliuoja paketų vidinius elementus, neleidžiant pakeitimais
išplisti už paketų ribų yra pagrindas tvirtai sistemos architektūrai, gebančiai efektyviai plėstis, ženkliai nesikeičiant ir sutaupant programos priežiūros kaštus.
