\section{Kompiuterinės sistemos vertinimas}

\subsection{Teisingai įgyvendinta kompiuterinė sistema}
Norint išsiaiškinti, kokią įtaką sistemos kokybei daro skirtingos paketų skirstymo metodologijos ir kaip ojektyviai tą įtaką pamatuoti, pirmiausia
reikėtų apsibrėžti kokiais požymiais pasižymi teisingai įgyvendinta kompiuterinė sistema.
Martin Kleppmann savo knygoje \textit{Designing Data-Intensive Applications: The Big Ideas Behind Reliable, Scalable, and Maintainable Systems} išskyria šiuos pagrindinius kriterijus:
\begin{itemize}
    \item Patikimumas, reiškiantis, kad net ir klaidų (įrangos, programinių ar žmogiškųjų) atveju,
    sistema veikia stabiliai ir patikimai, paslepiant tam tikras klaidas nuo vartotojo\cite{DataIntensiveApplications}.
    \item Prižiūrimumas, reiškiantis jog skirtingų abstrakcijų pagalba sumažintas sistemos kompleksiškumas.
    Dėl to nesunku keisti esamą sistemos funkcionalumą bei pritaikyti naujiems verslo naudojimo atvejams.
    Tai supaprastina darbą inžinierių ir operacijų komandoms dirbančioms su šia sistema, taip pat leidžia prie sistemos prisidėti naujiems žmonėms, o ne
    tik jos ekspertams.
    Tai ypač aktualu atviro kodo sistemoms\cite{DataIntensiveApplications}.
    \item Plečiamumas, reiškiantis jog sistema turi strategijas, kaip išlaikyti gerą našumą užklausų
    srautui didėjant ir sistemai augant, tai atliekant su pagrįstais kompiuteriniais resursais ir
    priežiūros kaina\cite{DataIntensiveApplications}.
\end{itemize}
Yra daug skirtingų elementų, sudarančių sistemą, kuri tenkintų aukščiau paminėtus kriterijus,
pavyzdžiui, pasirinktos technologijos, aukšto lygio architektūra, dokumentacija, sistemos testavimo
procesai, jų kiekis ir pan.
Vienas iš svarbių elementų, prisidedančių prie gerai įgyvendintos sistemos dizaino yra programinio kodo dizainas, jo skaitomumas, patikimumas.
Konvencijos, kaip vadinti kodo paketus, kokias klasės jiems priskirti ir kokios paketų hierarchijos laikytis sudaro svarbią programinio kodo dizaino dalį,
Todėl programinės įrangos kurimo metu, laikas skirtas rasti sistemai tinkamą paketų skirstymo šabloną ir to šablono laikymasis atsiperka, padarant
programinį kodą geriau suprantamu, taip prisidėdant prie bendro sistemos dizaino patikimumo ir lengvesnio palaikomumo.

Straipsnyje \textit{Investigating The Effect of Software Packaging on Modular Structure Stability}, jo autoriai akcentuoja, kad
gerai įgyvendintos, objektiškai orientuotos sistemos turėtų vystytis be didelių pakeitimu jų architektūroje.
To siekiama todėl, nes architektūriniai pakeitimai paveikią didelę sistemos dalį ir todėl
jų įgyvendinimo ir priežiūros kaštai yra ženkliai didesni\cite{ModularStability}.
Paketų struktūra, kuri užtikrina atsieta (\angl{decoupled}) komunikavima tarp paketų, enkapsuliuoja paketų vidinius elementus, neleidžiant pakeitimais
išplisti už paketų ribų yra pagrindas tvirtai sistemos architektūrai, gebančiai efektyviai plėstis, ženkliai nesikeičiant ir sutaupant programos priežiūros kaštus.

\subsection{Kodo skirstymo paketais metodų vertinimas}
Ankstesniame skyriuje buvo nagrinėjama gerai įgyvendintos paketų struktūros įtaka geram kompiuterinės sistemos dizainui,
tačiau lieka neatsakytas klausimas - kaip įvertinti metodą kodui į paketus grupuoti, kaip objektyiviai užtikrinti
jog šis sprendimas yra butent toks kokio reikia ir kokia būtent jo įtaka kompiuterinei sistemai?
Tvarkingas, aiškiai suprantamas kodas yra subjektyvi tema, priklausanti nuo komandos,
naudojamos programavimo kalbos ar programinių įrankių bei programinės sistemos dalykinės srities.
Kodo grupavimo į paketus metodai, taip, kaip ir bendros tvarkingo kodo praktikos,
gali būti labai subjektyvūs ir patogus tik metodą formavusiam asmeniui.
Tam, kad pagrįstai įvertinti skirtingus kodo skirstymo šablonus, pasiekiant kuo objektyvesnį,
plačiau priimtiną rezultatą, reikėtų aprašyti kriterijus, ko tikimasi iš paketų struktūros.

Robert C. Martin savo knygoje \textit{Agile Software Development, Principles, Patterns, and Practices} aprašo
šesis principus padedančius teisingai grupuoti klases į paketus.
Rodiklis, kiek kiekvienas paketas sistemoje laikosi nurodytų principų, tam tikrame paketų skirstymo šablone, gali
būti kriterijus, įvertinti to šablono kokybei ir įtakai bendram sistemos dizainui.

Pirmi trys principai yra apie paketo elementų sąryšį, jie padeda paskirstyti klases paketams\cite{AgileSoftwareDevelopment}.
Kiti trys principai reglamentuoja paketų sąjungą, jie padeda nustatyti, kaip paketai turėtų būti tarpusavyje susiję.

Paketų stabilumo metrikos:
\begin{itemize}
    \item \textit{Klasių skaičius} - klasių skaičiaus metrika paketui nurodo, skaičių klasių (konkrečių ir abstrakčių) pakete.
    Ši metrika matuoja paketo dydį.
    \item \textit{Aferentinės jungtys \angl{Afferent Couplings}} - aferentinių jungčių metrika nurodo
    skaičių kitų paketų, kurie priklauso nuo klasių esančių pasirinktame pakete.
    Ši metrika matuoja ateinančias priklausomybes.
    \item \textit{Eferentinės jungtys \angl{Efferent Couplings}} - eferentinių jungčių metrika nurodo skaičių kitų paketų,
    nuo kuriu priklauso klasės pasirintame pakete.
    Ši metrika matuoja išeinančias priklausomybės.
    \item \textit{Nestabilums} - nestabilumo metrika nurodo santyki tarp eferentinių jungčių ir
    visų jungčių (Aferentinės + Eferentinės) pakete.
    Ši metrika matuoja paketo atsparumui pokyčiams.
    Reikšmės rėžiai - nuo nulio iki vieno, kur vienas nurodo visiškai stabilų paketą, o vienetas visiškai nestabilų.
    \item \textit{Atstumas} - Atstumo metrika apibrėžiamas kaip statmenas pakuotės atstumas nuo idealizuotos linijos (A + I = 1),
    kur A yra abstrakčių klasių procentas nuo bendro paketo klasių skaičiaus.
    Ši metrika yra paketo abstraktumo ir stabilumo pusiausvyros rodiklis.
    Paketas tiesiai pagrindinėje sekoje yra optimaliai subalansuotas, atsižvelgiant į jos abstraktumą ir stabilumą.
    Šios metrikos diapazonas yra nuo nulio iki vieneto, o nulis nurodo paketą, kuris sutampa su pagrindine seka,
    o vienas – paketą, kuris yra kuo toliau nuo pagrindinės sekos. todo: pataisyti
    \item \textit{Žiedinės priklausomybės} - žiedinių priklausomybių metrika skaičių atveju, kur pasirinkto paketo išeinančios priklausomybes taip pat
    yra paketo ateinančios priklausomybes (tiesiogiai arba netiesiogiai)
\end{itemize}