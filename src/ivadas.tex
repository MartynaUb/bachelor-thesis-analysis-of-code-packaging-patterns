\sectionnonum{Įvadas}
\subsection{Problema ir jos aktualumas}
Teisingai įgyvendintas kompiuterinės sistemos dizainas yra vienas iš kritinių sėkmingo verslo
elementų.
Tam, jog verslas išlaikytų stabilų augimą yra būtina sukurti sistemą, kuri sumažintų
atotrūkį tarp organizacijos tikslų ir jų įgyvendinimo galimybių.
Mąstant apie programinio kodo
dizainą, kodo paketų kūrimas, klasių priskyrimas jiems ir paketų hierarchijos sudarymas paprastai
nėra pagrindinis prioritetas, tačiau tai parodo praleistą galimybę padaryti sistemos dizainą labiau
patikimu[Sho19], suprantamu [Eli10] ir lengviau palaikomu.
Modernios sistemos yra didžiulės, programinis kodas yra padalintas į daugybę failų,
kurie išskaidyti per skirtingo gylio direktorijas, todėl apgalvotai išskirstytas programinis
kodas daro daug didesnę įtaką kodo kokybei, nei gali atrodyti iš pirmo žvilgsnio.
Sistemos paketų studijavimas ir analizė norint įvertinti programinės įrangos kokybę
tampa vis svarbesne tema, dėl pastoviai augančio failų ir paketų skaičiaus\cite{DesignMetrics}.
Gerai įgyvendinta sistema turėtų gebėti keistis be didelių pakeitimų jos architektūroje,
to yra siekiama todėl, nes įgyvendinti architektūrinius pakeitimus kainuoja didelius kaštus, kurie
galėtų būti skirti produkto ir sistemos vystymuisi, o ne priežiūrai\cite{ModularStability}.
Būtent teisinga sistemos struktūra gali sumažinti, pakeitimų nutekėjimą už paketo vidaus, ir įtaką bendrai
sistemos architektūrai o ne vienam uždaram moduliui.


Norint išsiaiškinti, kaip efektyviausiai gali būti skaidomas programinis
kodas, tam jog jo struktūra darytų teigiama įtaką sistemos kokybei, reikalinga atlikti skirstymo į paketus šablonų analizę -
išsiaiškinti galimus šablonus ir būdus, kaip skirstyti programinį kodą į paketus, turėti aiškius šablonų apibrėžimus su jų
privalumai bei trūkumais.

\subsection{Darbo tikslas}
Šio darbo tikslas - identifikuoti šablonus kodo skirstymui į paketus.
Remiantis moksliniais straipsniais apie sistemos kokybę bei palaikomumą aprašyti kriterijus,
kurie būtų naudojami šablonus įvertinti, nustatant jų įtaką sistemos palaikomumui.

\subsection{Keliami uždaviniai}
\begin{itemize}
    \item Išskirti gerai įgyvendinto kodo požymius
    \item  Aprašyti skirstymo į paketus šablonus, remiantis pavyzdžiais teorinėje medžiagoje bei
egzistuojančiose atviro kodo sistemose
    \item  Įvertinti kiek realių sistemų struktūra nutolusi nuo teorinių šablonų apibrėžimų
    \item  Pasiūlyti kriterijus, įvertinančius kodo suskirstymo šablono įtaką sistemos kokybei, remiantis
rastai gerai įgyvenditos sistemos požymiais
    \item  Pasirinkti kelias sistemas ir pertvarkyti jų failų struktūrą pagal aprašytus šablonus, įvertinant
kiek sudėtinga pasiekti kiekvieno šablono strukūrą
    \item  Naudojant pertvarkytas sistemas, įvertinti kiekvieną kodo skirstymo šabloną pagal pasiūlytus kriterijus
    \item  Pateikti rekomendacijas, kokius šablonus kodo skirstymui tinkamiausia naudoti
\end{itemize}
todo: apubidinti šaltinius

\subsection{Numatomas darbo atlikimo procesas}
\begin{itemize}
    \item Remiantis teorine medžiaga aprašomi gerai įgyvendinto kodo požymiai, užtikrinantys
sistemos stabilumą ir palaikomumą.
    \item Aprašomi kriterijai, kuriuos naudojant galima įvertinti kodo suskirstymo įtaką sistemos
kodo kokybei, pavyzdžiui:
    \begin{itemize}
        \item Komponentų skaičius, priklausomas nuo pasirinkto komponento[Mah03]
        \item Tiesioginės ir netiesioginės priklausomybės (matomumas)[Ala07]
    \end{itemize}
    \item Galimų kodo skirstymo šablonų išskyrimas pasitelkiant teorinę informaciją [Sho11] ir
egzistuojančių sistemų architektūrą.
    \item Atviro kodo projektų pasirinkimas. Pasirenkami skirtingo tipo projektai, užtikrinant
objektyvesnę šablonų analizę skirtingose srityse. Galimi tipai:
    \begin{itemize}
        \item Taikomoji programinė įranga, teikianti paslaugas įrangos naudotojams. Pavyzdžiui,
internetinė programėlė priminimams ir darbams užsirašyti
        \item Techninė programinė įranga, naudojama taikomosios programinės įrangos duomenų
saugojimui, siuntimui, paieškai. Pavyzdžiui, duomenų bazės, pranešimų eilės, talpyklos
(angl. cache)
        \item Programinės įrangos įrankiai, skirti naudoti kitose sistemose supaprastinant programinį
kodą, naudojant jau įgyvendintas funkcijas. Pavyzdžiui, Java programavimo kalbos
Spring karkasas internetinių programėlių kūrimui
    \end{itemize}
    \item Pasirinktų projektų paketų sturktūros pertvarkymas pagal pasirinktus skirstymo šablonus
    \item Pertvarkytų projektų įvertinimas, naudojant išskirtus kriterijus, nustatant, kokią įtaką
skirtingi skirstymo šablonai turi sistemos kokybei
\end{itemize}


\subsection{Laukiami rezultatai}
\begin{itemize}
    \item Įdentifikuoti kodo skirstymo šablonai, remiantis teorine informacija ir praktikoje sutinkamais pavyzdžiais
    \item Sukurti kriterijai, įvertinantys sistemos paketų struktūros indėlį sistemos kokybei
    \item Pasirinkti projektai pertvarkyti pagal kodo skirstymo šablonus
    \item Įvertinus šablonus sukurtais kriterijais, pateiktos rekomendacijos kodo skirstymo šablonų naudojimui
\end{itemize}


\subsection{todo:}
aprašyti sekančius chapterius.