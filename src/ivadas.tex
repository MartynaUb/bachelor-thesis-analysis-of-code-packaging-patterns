\sectionnonum{Įvadas}
Gerai suprojektuota kompiuterinė sistema yra vienas iš kritinių sėkmingo verslo
elementų.
Tam, jog informacines sistemas naudojantis verslas išlaikytų stabilų augimą, yra būtina sukurti sistemą, kuri sumažintų
atotrūkį tarp organizacijos tikslų ir jų įgyvendinimo galimybių.
Mąstant apie programinio kodo
projektavimą, kodo paketų kūrimas, klasių priskyrimas jiems ir paketų hierarchijos sudarymas paprastai
nėra pagrindinis prioritetas, tačiau tai parodo praleistą galimybę padaryti sistemos struktūrą labiau
patikima [Sho19], suprantama [Eli10] ir lengviau palaikoma.
Modernios informacinės sistemos yra didžiulės, programinis kodas yra padalintas į daugybę failų,
kurie išskaidyti per skirtingo gylio direktorijas, todėl apgalvotai išskirstytas programinis
kodas daro daug didesnę įtaką bendram sistemos našumui, suprantamumui bei palaikomumui, nei gali atrodyti iš pirmo žvilgsnio.
Sistemos paketų struktūros analizė norint įvertinti programinės įrangos kokybę
tampa vis svarbesne tema dėl augančio failų ir paketų skaičiaus~\cite{DesignMetrics}.

Norint išsiaiškinti, kaip programinis kodas gali būti skirstomas efektyviausiai,
tam jog jo struktūra darytų teigiamą įtaką sistemai, reikalinga atlikti skirstymo į paketus šablonų analizę -
išsiaiškinti galimus šablonus, kaip skirstyti programinį kodą į paketus, turėti aiškius šablonų apibrėžimus su jų
privalumais bei trūkumais.
Šiame darbe minint \textit{šabloną kodo skirstymui į paketus} turima omenyje taisyklių arba metodų rinkinį,
nurodantį, kaip grupuoti klases į paketus, užtikrintant nuoseklų stilių bei sprendžiant iškilusią problemą.

Šio darbo tikslas - identifikuoti ir įvertinti praktikoje naudojamus šablonus kodo skirstymui į paketus.
Tai atliekama remiantis moksliniais straipsniais apie sistemos kokybę bei palaikomumą, aprašant kriterijus,
kurie būtų naudojami šablonams įvertinti, nustatant jų įtaką sistemos palaikomumui, patikimumui bei plečiamumui.

Tikslui pasiekti yra iškeliami šie uždaviniai:
\begin{itemize}
    \item  Išskirti gerai įgyvendinto kodo požymius
    \item  Aprašyti skirstymo į paketus šablonus, remiantis praktikoje sutinkamais pavyzdžiais
    \item  Pasiūlyti kriterijus, įvertinančius kodo suskirstymo šablono įtaką sistemos kokybei, remiantis
rastais gerai įgyvendintos sistemos požymiais
    \item  Pasirinkti kelias sistemas ir pertvarkyti jų failų struktūrą pagal aprašytus šablonus, įvertinant,
kiek sudėtinga pasiekti kiekvieno šablono strukūrą
    \item  Naudojant pertvarkytas sistemas, įvertinti kiekvieną kodo skirstymo šabloną pagal pasiūlytus kriterijus
    \item  Pateikti rekomendacijas, ar aprašyti šablonai kodo skirstymui tinkami naudoti
\end{itemize}

Šio darbo metu nagrinėjami ir aprašomi gerai įgyvendinto kodo požymiai, užtikrinantys
sistemos stabilumą ir palaikomumą, remiantis Martin Kleppmann \textit{Designing Data-Intensive Applications: The Big Ideas Behind Reliable, Scalable, and Maintainable Systems},
ir Robert C. Martin \textit{Agile Software Development, Principles, Patterns, and Practices} knygomis.
Ieškomi kriterijai, kuriuos naudojant galima įvertinti kodo suskirstymo įtaką sistemos kokybei, pavyzdžiui - komponentų skaičius,
tiesioginės ir netiesioginės priklausomybės, paketų stabilumas~\cite{AgileSoftwareDevelopment}.
Tyrinėjami šablonai kodo skirstymui į paketus įvardinti Martin Sadin straipsnyje \textit{Four Strategies for Organizing Code}.
Nagrinėjamos atviro kodo sistemos, pasirenkant skirtingo tipo projektus, siekiant
objektyvesnės šablonų analizės skirtingose srityse.
Galimi tipai:
    \begin{itemize}
        \item Taikomoji programinė įranga, teikianti paslaugas įrangos naudotojams. Pavyzdžiui,
internetinė programėlė priminimams ir darbams užsirašyti
        \item Techninė programinė įranga, naudojama taikomosios programinės įrangos duomenų
saugojimui, siuntimui, paieškai. Pavyzdžiui, duomenų bazės, pranešimų eilės, talpyklos
(angl. cache)
        \item Programinės įrangos įrankiai, skirti naudoti kitose sistemose supaprastinant programinį
kodą, naudojant jau įgyvendintas funkcijas. Pavyzdžiui, Java programavimo kalbos
Spring karkasas internetinių programėlių kūrimui
    \end{itemize}
Tyrinėjamų projektų paketų sturktūros pertvarkomos pagal pasirinktus skirstymo šablonus, pertvarkyti projektai įvertinti, naudojant išskirtus kriterijus, nustatant, kokią įtaką
skirtingi skirstymo šablonai turi sistemos kokybei.

Likusi šio dokumento dalis yra išdėstyta taip - pirmas skyrius nagrinėja tvarkingos kompiuterinės sistemos sąvoką, kas ją sudaro, įgyvendinimo kokybę ir kaip galima ją įvertinti.
Aprašyti kriterijai, kaip įvertinti paketų struktūros įtaką sistemos kokybei.
Antras skyrius tyrinėja skirtingus šablonus klasėms į paketus skirstyti, jų privalumus bei trūkumus.
Trečiame skyriuje aprašomi sukurti įrankiai, reikalingi sistemų analizei ir šablonų įvertinimui, minima, kaip jie įgyvendinti ir kaip jie yra naudojami.
Ketvirtame skyriuje analizuojamos pasirinktos atviro kodo sistemos - bandoma nustatyti jų naudojamus šablonus, vertinama sistemų kokybė.
Penktame skyriuje aprašomas procesas, kaip pasirinktos sistemos yra perdaromos, kad tiksliai laikytųsi antrame skyriuje aprašytų kodo skirstymų šablonų, įvertinama, kiek sudėtinga pasiekti kiekvieno šablono struktūrą.
Nagrinėjama perdarytų sistemų kokybė pagal pirmame skyriuje aprašytus kriterijus, ieškomi šablonų privalumai bei trūkumai.
