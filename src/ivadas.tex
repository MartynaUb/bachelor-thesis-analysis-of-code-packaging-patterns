\sectionnonum{Įvadas}
\subsection{Problema ir jos aktualumas}
Teisingai įgyvendintas kompiuterinės sistemos dizainas yra vienas iš kritinių sėkmingo verslo
elementų.
Tam, jog verslas išlaikytų stabilų augimą, yra būtina sukurti sistemą, kuri sumažintų
atotrūkį tarp organizacijos tikslų ir jų įgyvendinimo galimybių.
Mąstant apie programinio kodo
dizainą, kodo paketų kūrimas, klasių priskyrimas jiems ir paketų hierarchijos sudarymas paprastai
nėra pagrindinis prioritetas, tačiau tai parodo praleistą galimybę padaryti sistemos dizainą labiau
patikimu[Sho19], suprantamu[Eli10] ir lengviau palaikomu.
Modernios sistemos yra didžiulės, programinis kodas yra padalintas į daugybę failų,
kurie išskaidyti per skirtingo gylio direktorijas, todėl apgalvotai išskirstytas programinis
kodas daro daug didesnę įtaką kodo kokybei, nei gali atrodyti iš pirmo žvilgsnio.
Sistemos paketų studijavimas ir analizė norint įvertinti programinės įrangos kokybę
tampa vis svarbesne tema dėl augančio failų ir paketų skaičiaus\cite{DesignMetrics}.

Norint išsiaiškinti, kaip efektyviausiai gali būti skaidomas programinis
kodas, tam jog jo struktūra darytų teigiama įtaką sistemos kokybei, reikalinga atlikti skirstymo į paketus šablonų analizę -
išsiaiškinti galimus šablonus, kaip skirstyti programinį kodą į paketus, turėti aiškius šablonų apibrėžimus su jų
privalumais bei trūkumais.
Šiame darbe minint \textit{šabloną kodo skirstymui į paketus} turima omeny planą arba metodų rinkinys, nurodantį, kaip grupuoti klases į paketus, užtikrintant nuoseklų stilių.


Šio darbo tikslas - identifikuoti ir ivertinti šablonus kodo skirstymui į paketus.
Remiantis moksliniais straipsniais apie sistemos kokybę bei palaikomumą aprašyti kriterijus,
kurie būtų naudojami šablonus įvertinti, nustatant jų įtaką sistemos palaikomumui.

Tikslui pasiekti yra iškeliami šie uždaviniai:
\begin{itemize}
    \item Išskirti gerai įgyvendinto kodo požymius
    \item  Aprašyti skirstymo į paketus šablonus, remiantis pavyzdžiais teorinėje medžiagoje
    \item  Įvertinti kiek realių sistemų struktūra nutolusi nuo teorinių šablonų apibrėžimų
    \item  Pasiūlyti kriterijus, įvertinančius kodo suskirstymo šablono įtaką sistemos kokybei, remiantis
rastai gerai įgyvenditos sistemos požymiais
    \item  Pasirinkti kelias sistemas ir pertvarkyti jų failų struktūrą pagal aprašytus šablonus, įvertinant
kiek sudėtinga pasiekti kiekvieno šablono strukūrą
    \item  Naudojant pertvarkytas sistemas, įvertinti kiekvieną kodo skirstymo šabloną pagal pasiūlytus kriterijus
    \item  Pateikti rekomendacijas, kokius šablonus kodo skirstymui tinkamiausia naudoti
\end{itemize}

Šiuo darbo metu nagrinėjami ir aprašomi gerai įgyvendinto kodo požymiai, užtikrinantys
sistemos stabilumą ir palaikomumą, remiantis Martin Kleppmann \textit{Designing Data-Intensive Applications: The Big Ideas Behind Reliable, Scalable, and Maintainable Systems},
ir Robert C. Martin \textit{Agile Software Development, Principles, Patterns, and Practices} knygomis.
Ieškomi kriterijai, kuriuos naudojant galima įvertinti kodo suskirstymo įtaką sistemos kodo kokybei, tokie kaip - komponentų skaičius, tiesioginės ir netiesioginės priklausomybės, paketų stabilumas\cite{AgileSoftwareDevelopment}.
Tyrinėjami šablonai kodo skirstymui į paketus įvardinti Martin Sadin straipsnyje \textit{Four Strategies for Organizing Code}.
Pasirintos ir išnagrinėtos atviro kodo sistemos.
Pasirenkant skirtingo tipo projektus, siekiant
objektyvesnės šablonų analizės skirtingose srityse.
Galimi tipai:
    \begin{itemize}
        \item Taikomoji programinė įranga, teikianti paslaugas įrangos naudotojams. Pavyzdžiui,
internetinė programėlė priminimams ir darbams užsirašyti
        \item Techninė programinė įranga, naudojama taikomosios programinės įrangos duomenų
saugojimui, siuntimui, paieškai. Pavyzdžiui, duomenų bazės, pranešimų eilės, talpyklos
(angl. cache)
        \item Programinės įrangos įrankiai, skirti naudoti kitose sistemose supaprastinant programinį
kodą, naudojant jau įgyvendintas funkcijas. Pavyzdžiui, Java programavimo kalbos
Spring karkasas internetinių programėlių kūrimui
    \end{itemize}
Tyrinėjamų projektų paketų sturktūros pertvarkomos pagal pasirinktus skirstymo šablonus, pertvarkyti projektai įvertinti, naudojant išskirtus kriterijus, nustatant, kokią įtaką
skirtingi skirstymo šablonai turi sistemos kokybei.

Iš šio darbo yra laukiami šie rezultatai:
\begin{itemize}
    \item Įdentifikuoti kodo skirstymo šablonai, remiantis teorine informacija
    \item Sukurti kriterijai, įvertinantys sistemos paketų struktūros indėlį sistemos kokybei
    \item Pasirinkti projektai pertvarkyti pagal kodo skirstymo šablonus
    \item Įvertinus šablonus sukurtais kriterijais, pateiktos rekomendacijos kodo skirstymo šablonų naudojimui
\end{itemize}


Likusi šio dokumento dalis yra išdėstyta taip - pirmas skyrius nagrinėja tvarkingos kompiuterinės sistemos sąvoka, kas ją sudaro, kaip galima įvertinti sistemos tvarkingumą, įgyvendinimo teisingumą.
Aprašyti kriterijai kaip įvertinti paketų struktūros įtaka, sistemos kokybei.
Antras skyrius tyrinėja skirtingus šablonųs klasėms į paketus skirstyti, jų privalumus bei trūkumus.
Trečiame skyriuje aprašomi sukurti įrankiai, reikalingį sistemų analizei ir šablonų įvertinimui, minima kaip jie įgyvendinti ir kaip jie yra naudojami.
Ketvirtame skyriuje analizuojamos pasirinktos atviro kodo sistemos - bandoma nustatyti jų naudojamus šablonus, vertinama sistemų kokybė.
Penktame skyriuje aprašomas procesas, kaip pasirinktos sistemos yra perdaromos, kad tiksliai laikytųsi antrame skyriuje aprašytų kodo skirstymų šablonų, įvertinama kiek sudėtingą pasiekti kiekvieno šablono struktūrą.
Nagrinėjama perdarytų sistemų kokybė pagal pirmame skyriuje aprašytus kriterijus, ieškomas geriausiai įvertintas šablonas.
Straipsnis baigiamas šeštu skyriumi su išvadomis bei gautų rezultatų analize.