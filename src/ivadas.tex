\sectionnonum{Įvadas}
Gerai suprojektuota kompiuterinė sistema yra vienas iš kritinių sėkmingo ją naudojančio verslo
elementų.
Tam, jog kompiuterines sistemas naudojantis verslas išlaikytų stabilų augimą, jo sistema turėtų kiek įmanoma sumažinti
atotrūkį tarp organizacijos tikslų ir jų įgyvendinimo galimybių.
Mąstant apie programinio kodo
projektavimą, paketų kūrimas, klasių priskyrimas jiems ir paketų hierarchijos sudarymas paprastai
nėra pagrindinis prioritetas, tačiau tai parodo praleistą galimybę padaryti sistemos struktūrą labiau
patikima, suprantama ir lengviau palaikoma.
Modernios kompiuterinės sistemos yra didžiulės, programinis kodas yra padalintas į daugybę failų,
kurie yra išskaidyti per skirtingo gylio direktorijas, todėl apgalvotai išskirstytas programinis
kodas daro didesnę įtaką bendram sistemos suprantamumui bei palaikomumui, nei gali atrodyti iš pirmo žvilgsnio.
Sistemos paketų struktūros analizė norint įvertinti programinės įrangos kokybę
tampa vis svarbesne tema dėl augančio failų ir paketų skaičiaus~\cite{DesignMetrics}.

Diskusijose, kaip reikėtų skirstyti programinį kodą, paprastai akcentuojami du metodai - pagal \textit{techninį funkcionalumą},
kur kiekvienai techninei funckijai arba kompiuterinės sistemos sluoksniui yra sukuriamas paketas,
arba pagal \textit{dalykinės srities esybes}, kur vienos esybės kodas, dalykinės srities
esybės funkcionalumas skirtingose programiniuose sluoksniuose yra patalpintas viename pakete~\cite{PackagingWays}.
Dažniausiai, prioritetas teikiamas kodo skirstymui pagal dalykinės srities esybes~\cite{DomainDrivenDesign} dėl lengvesnio skaitomumo,
palaikomumo, mažesnės sankibos~\cite{DivideByDomain}.
Skirstymas pagal dalykinės srities esybes taip pat gali padidinti
inkapsuliaciją naudojant prieigos modifikatorius \textit{Java} programavimo kalboje.

Tačiau šis metodas negali būti vienareikšmiškai pritaikomas kiekvienoje situacijoje ir dažniausiai praktikoje jo nėra griežtai laikomasi -
klasės būna išskirstytos remiantis papildomomis taisyklėmis,
siekiant išspręsti sistemos planavimo metu kylančias problemas.
Kadangi nėra aiškiai apibrėžtų gairių, kaip skirstymo metodas gali būti pritaikomas kilus sudėtingesnei situacijai, praktikoje pasitaiko skirtingų
jų sprendimo būdų, galinčių neigiamai paveikti sistemos kokybę.
Norint išsiaiškinti, koks programinio kodo skirstymo būdas tinkamiausias tokiose situacijose,
tam jog jo struktūra darytų teigiamą įtaką sistemai, reikalinga atlikti skirstymo į paketus šablonų analizę -
išsiaiškinti galimus šablonus, naudojamus kylančių projektavimo problemų sprendimui, turėti aiškius šablonų apibrėžimus su jų
privalumais bei trūkumais.
Šio darbo metu nagrinėjami ir aprašomi gerai įgyvendinto kodo požymiai, užtikrinantys
sistemos stabilumą ir palaikomumą, remiantis Martin Kleppmann \textit{Designing Data-Intensive Applications: The Big Ideas Behind Reliable, Scalable, and Maintainable Systems},
ir Robert C. Martin \textit{Agile Software Development, Principles, Patterns, and Practices} knygomis.
Šiame darbe minint \textit{kodo skirstymo į paketus šabloną} turima omenyje taisykles,
nurodančias, kaip grupuoti klases į paketus, sprendžiant iškilusią problemą bei užtikrintant nuoseklų stilių.

Šio darbo tikslas - identifikuoti ir įvertinti praktikoje naudojamus šablonus kodo skirstymui į paketus.

Tikslui pasiekti yra iškeliami šie uždaviniai:
\begin{itemize}
    \item  Išskirti gerai įgyvendinto kodo požymius
    \item  Aprašyti skirstymo į paketus šablonus, remiantis praktikoje sutinkamais pavyzdžiais
    \item  Pasiūlyti kriterijus, įvertinančius kodo skirstymo šablono įtaką sistemos kokybei, remiantis
rastais gerai įgyvendintos sistemos požymiais
    \item  Pasirinkti kelias sistemas ir pertvarkyti jų failų struktūrą pagal aprašytus šablonus, įvertinant,
kiek sudėtinga pasiekti kiekvieno šablono strukūrą
    \item  Naudojant pertvarkytas sistemas, įvertinti kiekvieną šabloną kodo skirstymui į paketus pagal pasiūlytus kriterijus
    \item  Pateikti rekomendacijas, ar aprašyti šablonai kodo skirstymui tinkami naudoti
\end{itemize}




