\sectionnonum{Įvadas}
Gerai suprojektuota kompiuterinė sistema yra vienas iš kritinių sėkmingo verslo
elementų.
Tam, jog informacines sistemas naudojantis verslas išlaikytų stabilų augimą, yra būtina sukurti sistemą, kuri sumažintų
atotrūkį tarp organizacijos tikslų ir jų įgyvendinimo galimybių.
Mąstant apie programinio kodo
projektavimą, kodo paketų kūrimas, klasių priskyrimas jiems ir paketų hierarchijos sudarymas paprastai
nėra pagrindinis prioritetas, tačiau tai parodo praleistą galimybę padaryti sistemos struktūrą labiau
patikima [Sho19], suprantama [Eli10] ir lengviau palaikoma.
Modernios informacinės sistemos yra didžiulės, programinis kodas yra padalintas į daugybę failų,
kurie išskaidyti per skirtingo gylio direktorijas, todėl apgalvotai išskirstytas programinis
kodas daro daug didesnę įtaką bendram sistemos našumui, suprantamumui bei palaikomumui, nei gali atrodyti iš pirmo žvilgsnio.
Sistemos paketų struktūros analizė norint įvertinti programinės įrangos kokybę
tampa vis svarbesne tema dėl augančio failų ir paketų skaičiaus~\cite{DesignMetrics}.

Diskusijose, kaip reikėtų skirstyti programinį kodą, paprastai akcentuojami du metodai - pagal \textit{techninį sluoksnį},
kur kiekvienam funkcionalumui arba kompiuterinės sistemos sluoksniui yra sukuriamas paketas,
arba pagal \textit{dalykinės srities esybes}, kur vienos esybės kodas, dalykinės srities
esybės funkcionalumas skirtingose programiniuose sluoksniuose yra patalpintas viename pakete~\cite{PackagingWays}.
Dažniausiai, prioritetas teikiamas kodo skirstymui pagal dalykinės srities esybes~\cite{DomainDrivenDesign} dėl lengvesnio skaitomumo,
palaikomumo, mažesnės sankibos~\cite{DivideByDomain}. Skirstymas pagal dalykinės srities esybes taip pat gali padidinti
enkapsuliaciją naudojant prieigos modifikatorius \textit{Java} programavimo kalboje.

Tačiau šis metodas negali būti vienareikšmiškai pritaikomas kiekvienoje situacijoje ir, dažniausiai praktikoje jo nėra griežtai laikomasi -
klasės būna išskaidytos remiantis papildomomis taisyklėmis,
siekiant išspręsti sistemos planavimo metu kylančias problemas.
Kadangi nėra aiškiai apibrėžtų gairių, kaip skirstymo metodas gali būti pritaikomas kilus sudėtingesnei situacijai, praktikoje pasitaiko skirtingų
jų sprendimo būdų, galinčių neigiamai paveikti sistemos kokybę.
Norint išsiaiškinti, kaip programinis kodas gali būti skirstomas efektyviausiai tokiose situacijose,
tam jog jo struktūra darytų teigiamą įtaką sistemai, reikalinga atlikti skirstymo į paketus šablonų analizę -
išsiaiškinti galimus šablonus, naudojamus kylančių projektavimo problemų sprendimui, turėti aiškius šablonų apibrėžimus su jų
privalumais bei trūkumais.
Šiame darbe minint \textit{šabloną kodo skirstymui į paketus} turima omenyje taisykles,
nurodančias, kaip grupuoti klases į paketus, sprendžiant iškilusią problemą bei užtikrintant nuoseklų stilių.

Šio darbo tikslas - identifikuoti ir įvertinti praktikoje naudojamus šablonus kodo skirstymui į paketus.

Tikslui pasiekti yra iškeliami šie uždaviniai:
\begin{itemize}
    \item  Išskirti gerai įgyvendinto kodo požymius
    \item  Aprašyti skirstymo į paketus šablonus, remiantis praktikoje sutinkamais pavyzdžiais
    \item  Pasiūlyti kriterijus, įvertinančius kodo suskirstymo šablono įtaką sistemos kokybei, remiantis
rastais gerai įgyvendintos sistemos požymiais
    \item  Pasirinkti kelias sistemas ir pertvarkyti jų failų struktūrą pagal aprašytus šablonus, įvertinant,
kiek sudėtinga pasiekti kiekvieno šablono strukūrą
    \item  Naudojant pertvarkytas sistemas, įvertinti kiekvieną šabloną kodo skirstymui į paketus pagal pasiūlytus kriterijus
    \item  Pateikti rekomendacijas, ar aprašyti šablonai kodo skirstymui tinkami naudoti
\end{itemize}

Šio darbo metu nagrinėjami ir aprašomi gerai įgyvendinto kodo požymiai, užtikrinantys
sistemos stabilumą ir palaikomumą, remiantis Martin Kleppmann \textit{Designing Data-Intensive Applications: The Big Ideas Behind Reliable, Scalable, and Maintainable Systems},
ir Robert C. Martin \textit{Agile Software Development, Principles, Patterns, and Practices} knygomis.
Ieškomi kriterijai, kuriuos naudojant galima įvertinti kodo suskirstymo įtaką sistemos kokybei, pavyzdžiui - komponentų skaičius,
tiesioginės ir netiesioginės priklausomybės, paketų stabilumas~\cite{AgileSoftwareDevelopment}.
Nagrinėjamos atviro kodo sistemos, pasirenkant skirtingo tipo projektus, siekiant
objektyvesnės šablonų analizės skirtingose srityse.
Galimi tipai:
    \begin{itemize}
        \item Taikomoji programinė įranga, teikianti paslaugas įrangos naudotojams. Pavyzdžiui,
internetinė programėlė priminimams ir darbams užsirašyti
        \item Techninė programinė įranga, naudojama taikomosios programinės įrangos duomenų
saugojimui, siuntimui, paieškai. Pavyzdžiui, duomenų bazės, pranešimų eilės, talpyklos
(angl. cache)
        \item Programinės įrangos įrankiai, skirti naudoti kitose sistemose supaprastinant programinį
kodą, naudojant jau įgyvendintas funkcijas. Pavyzdžiui, Java programavimo kalbos
Spring karkasas internetinių programėlių kūrimui
    \end{itemize}
Tyrinėjamų projektų paketų sturktūros pertvarkomos pagal pasirinktus skirstymo šablonus, pertvarkyti projektai įvertinti, naudojant išskirtus kriterijus, nustatant, kokią įtaką
skirtingi skirstymo šablonai turi sistemos kokybei.

Likusi šio dokumento dalis yra išdėstyta taip:
Pirmas skyrius tyrinėja skirtingus šablonus klasėms į paketus skirstyti, jų privalumus bei trūkumus.
Antras skyrius nagrinėja tvarkingos kompiuterinės sistemos sąvoką, kas ją sudaro, įgyvendinimo kokybę ir kaip galima ją įvertinti.
Aprašyti kriterijai, kaip įvertinti paketų struktūros įtaką sistemos kokybei.
Trečiame skyriuje aprašomi sukurti įrankiai, reikalingi sistemų analizei ir šablonų įvertinimui, minima, kaip jie įgyvendinti ir kaip jie yra naudojami.
Ketvirtame skyriuje analizuojamos pasirinktos atviro kodo sistemos - bandoma nustatyti jų naudojamus šablonus, vertinama sistemų kokybė.
Penktame skyriuje aprašomas procesas, kaip pasirinktos sistemos yra perdaromos, kad tiksliai laikytųsi antrame skyriuje aprašytų kodo skirstymų šablonų, įvertinama, kiek sudėtinga pasiekti kiekvieno šablono struktūrą.
Nagrinėjama perdarytų sistemų kokybė pagal pirmame skyriuje aprašytus kriterijus, vertinami šablonų privalumai bei trūkumai.
