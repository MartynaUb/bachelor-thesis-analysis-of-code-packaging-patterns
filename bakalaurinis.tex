\documentclass[
    % english, % Klasei padavus parametrą 'english', darbas bus anglų kalba.
    % signatureplaces % prideda parašų vietas tituliniame puslapyje
]{VUMIFPSbakalaurinis}
\usepackage{float}
\usepackage{wrapfig2}
\usepackage{hyperref}
\usepackage{algorithmicx}
\usepackage{algorithm}
\usepackage{algpseudocode}
\usepackage{amsfonts}
\usepackage{amsmath}
\usepackage{bm}
\usepackage{caption}
\usepackage{color}
\usepackage{graphicx}
\usepackage{listings}
\usepackage{subcaption}
\usepackage{biblatex}
\usepackage{framed,color,verbatim}
\usepackage{dirtree}
\usepackage[table]{xcolor}% http://ctan.org/pkg/xcolor
\definecolor{shadecolor}{rgb}{.9, .9, .9}
\usepackage{tabularx}
\newcolumntype{L}{>{\raggedright\arraybackslash}X}


% Titulinio aprašas
\university{Vilniaus universitetas}
\faculty{Matematikos ir informatikos fakultetas}
\department{Programų sistemų bakalauro studijų programa}
\papertype{Bakalauro baigiamasis darbas}
\title{Kodo skirstymo į paketus šablonų tyrimas}
\titleineng{Analysis of code packaging patterns}
\author{Martyna Ubartaitė}
\supervisor{Gediminas Rimša}
\reviewer{doc. dr. Audronė Lupeikienė}
\date{Vilnius – \the\year}

\bibliography{bibliografija}

\begin{document}
\maketitle

%% Padėkų skyrius
% \sectionnonumnocontent{}
% \vspace{7cm}
% \begin{center}
%     Padėkos asmenims ir/ar organizacijoms
% \end{center}

\begin{lithuanian}
\sectionnonumnocontent{Santrauka}
Gerai suprojektuota kompiuterinė sistema yra vienas iš kritinių sėkmingo ją naudojančio verslo
elementų.
Mąstant apie programinio kodo
projektavimą, kodo paketų kūrimas, klasių priskyrimas jiems ir paketų hierarchijos sudarymas paprastai
nėra pagrindinis prioritetas, tačiau tai parodo praleistą galimybę padaryti sistemos struktūrą labiau
stabilia, suprantama ir lengviau palaikoma.
Rekomenduojamas naudoti kodo skirstymo į paketus pagal dalykinę sritį metodas negali būti vienareikšmiškai pritaikomas
kiekvienoje situacijoje ir dažniausiai praktikoje jo nėra griežtai laikomasi - klasės būna išskaidytos remiantis papildomomis taisyklėmis,
siekiant išspręsti sistemos planavimo metu kylančias problemas.
Darbo metu išskirti planavimo metu kylančias problemas sprendžiantys kodo skirstymo į paketus šablonai, aprašyti sistemos
kokybę vertinantys matai.
Pagal rastus šablonus pertvarkytos atviro kodo sistemos, įvertinti pertvarkytų sistemų kokybės matai.
Darbo rezultatuose išskirti šablonai galintys papildyti kodo skirstymo į paketus pagal dalykinę sritį metodą ir spręsti
projektavimo metu kylančias problemas.

\raktiniaizodziai{Kodo skirstymas į paketus, Kodo skirstymo į paketus šablonai, Java, Paketai, Šablonai}
\end{lithuanian}

\begin{english}
\sectionnonumnocontent{Summary}
Creating packages and package hierarchy is usually not a top priority in software design, but
a correct approach to code packaging can make the system more readable, maintainable and improve the overall quality.
The approach to divide code into packages by functionality may not be applicable in every situation.
In an attempt to solve these issues, multiple code packaging patterns are used in real-world systems.
This paper attempts to identify such issues together with patterns which solve them.
To achieve this, a set of metrics evaluating readability, maintainability and stability were proposed.
Identified patters were verfified against chosen metrics to evaluate their impact.

\keywords{Code packaging, Code packaging patterns, Java, Packaging, Patterns}
\end{english}

\tableofcontents

\sectionnonum{Įvadas}
\subsection{Problema ir jos aktualumas}
Teisingai įgyvendintas kompiuterinės sistemos dizainas yra vienas iš kritinių sėkmingo verslo
elementų.
Tam, jog verslas išlaikytų stabilų augimą, yra būtina sukurti sistemą, kuri sumažintų
atotrūkį tarp organizacijos tikslų ir jų įgyvendinimo galimybių.
Mąstant apie programinio kodo
dizainą, kodo paketų kūrimas, klasių priskyrimas jiems ir paketų hierarchijos sudarymas paprastai
nėra pagrindinis prioritetas, tačiau tai parodo praleistą galimybę padaryti sistemos dizainą labiau
patikimu[Sho19], suprantamu[Eli10] ir lengviau palaikomu.
Modernios sistemos yra didžiulės, programinis kodas yra padalintas į daugybę failų,
kurie išskaidyti per skirtingo gylio direktorijas, todėl apgalvotai išskirstytas programinis
kodas daro daug didesnę įtaką kodo kokybei, nei gali atrodyti iš pirmo žvilgsnio.
Sistemos paketų studijavimas ir analizė norint įvertinti programinės įrangos kokybę
tampa vis svarbesne tema dėl augančio failų ir paketų skaičiaus\cite{DesignMetrics}.

Norint išsiaiškinti, kaip efektyviausiai gali būti skaidomas programinis
kodas, tam jog jo struktūra darytų teigiama įtaką sistemos kokybei, reikalinga atlikti skirstymo į paketus šablonų analizę -
išsiaiškinti galimus šablonus, kaip skirstyti programinį kodą į paketus, turėti aiškius šablonų apibrėžimus su jų
privalumais bei trūkumais.
Šiame darbe minint \textit{šabloną kodo skirstymui į paketus} turima omeny planą arba metodų rinkinys, nurodantį, kaip grupuoti klases į paketus, užtikrintant nuoseklų stilių.


Šio darbo tikslas - identifikuoti ir ivertinti šablonus kodo skirstymui į paketus.
Remiantis moksliniais straipsniais apie sistemos kokybę bei palaikomumą aprašyti kriterijus,
kurie būtų naudojami šablonus įvertinti, nustatant jų įtaką sistemos palaikomumui.

Tikslui pasiekti yra iškeliami šie uždaviniai:
\begin{itemize}
    \item Išskirti gerai įgyvendinto kodo požymius
    \item  Aprašyti skirstymo į paketus šablonus, remiantis pavyzdžiais teorinėje medžiagoje
    \item  Įvertinti kiek realių sistemų struktūra nutolusi nuo teorinių šablonų apibrėžimų
    \item  Pasiūlyti kriterijus, įvertinančius kodo suskirstymo šablono įtaką sistemos kokybei, remiantis
rastai gerai įgyvenditos sistemos požymiais
    \item  Pasirinkti kelias sistemas ir pertvarkyti jų failų struktūrą pagal aprašytus šablonus, įvertinant
kiek sudėtinga pasiekti kiekvieno šablono strukūrą
    \item  Naudojant pertvarkytas sistemas, įvertinti kiekvieną kodo skirstymo šabloną pagal pasiūlytus kriterijus
    \item  Pateikti rekomendacijas, kokius šablonus kodo skirstymui tinkamiausia naudoti
\end{itemize}

Šiuo darbo metu nagrinėjami ir aprašomi gerai įgyvendinto kodo požymiai, užtikrinantys
sistemos stabilumą ir palaikomumą, remiantis Martin Kleppmann \textit{Designing Data-Intensive Applications: The Big Ideas Behind Reliable, Scalable, and Maintainable Systems},
ir Robert C. Martin \textit{Agile Software Development, Principles, Patterns, and Practices} knygomis.
Ieškomi kriterijai, kuriuos naudojant galima įvertinti kodo suskirstymo įtaką sistemos kodo kokybei, tokie kaip - komponentų skaičius, tiesioginės ir netiesioginės priklausomybės, paketų stabilumas\cite{AgileSoftwareDevelopment}.
Tyrinėjami šablonai kodo skirstymui į paketus įvardinti Martin Sadin straipsnyje \textit{Four Strategies for Organizing Code}.
Pasirintos ir išnagrinėtos atviro kodo sistemos.
Pasirenkant skirtingo tipo projektus, siekiant
objektyvesnės šablonų analizės skirtingose srityse.
Galimi tipai:
    \begin{itemize}
        \item Taikomoji programinė įranga, teikianti paslaugas įrangos naudotojams. Pavyzdžiui,
internetinė programėlė priminimams ir darbams užsirašyti
        \item Techninė programinė įranga, naudojama taikomosios programinės įrangos duomenų
saugojimui, siuntimui, paieškai. Pavyzdžiui, duomenų bazės, pranešimų eilės, talpyklos
(angl. cache)
        \item Programinės įrangos įrankiai, skirti naudoti kitose sistemose supaprastinant programinį
kodą, naudojant jau įgyvendintas funkcijas. Pavyzdžiui, Java programavimo kalbos
Spring karkasas internetinių programėlių kūrimui
    \end{itemize}
Tyrinėjamų projektų paketų sturktūros pertvarkomos pagal pasirinktus skirstymo šablonus, pertvarkyti projektai įvertinti, naudojant išskirtus kriterijus, nustatant, kokią įtaką
skirtingi skirstymo šablonai turi sistemos kokybei.

Iš šio darbo yra laukiami šie rezultatai:
\begin{itemize}
    \item Įdentifikuoti kodo skirstymo šablonai, remiantis teorine informacija
    \item Sukurti kriterijai, įvertinantys sistemos paketų struktūros indėlį sistemos kokybei
    \item Pasirinkti projektai pertvarkyti pagal kodo skirstymo šablonus
    \item Įvertinus šablonus sukurtais kriterijais, pateiktos rekomendacijos kodo skirstymo šablonų naudojimui
\end{itemize}


Likusi šio dokumento dalis yra išdėstyta taip - pirmas skyrius nagrinėja tvarkingos kompiuterinės sistemos sąvoka, kas ją sudaro, kaip galima įvertinti sistemos tvarkingumą, įgyvendinimo teisingumą.
Aprašyti kriterijai kaip įvertinti paketų struktūros įtaka, sistemos kokybei.
Antras skyrius tyrinėja skirtingus šablonųs klasėms į paketus skirstyti, jų privalumus bei trūkumus.
Trečiame skyriuje aprašomi sukurti įrankiai, reikalingį sistemų analizei ir šablonų įvertinimui, minima kaip jie įgyvendinti ir kaip jie yra naudojami.
Ketvirtame skyriuje analizuojamos pasirinktos atviro kodo sistemos - bandoma nustatyti jų naudojamus šablonus, vertinama sistemų kokybė.
Penktame skyriuje aprašomas procesas, kaip pasirinktos sistemos yra perdaromos, kad tiksliai laikytųsi antrame skyriuje aprašytų kodo skirstymų šablonų, įvertinama kiek sudėtingą pasiekti kiekvieno šablono struktūrą.
Nagrinėjama perdarytų sistemų kokybė pagal pirmame skyriuje aprašytus kriterijus, ieškomas geriausiai įvertintas šablonas.
Straipsnis baigiamas šeštu skyriumi su išvadomis bei gautų rezultatų analize.
\section{Galimi kodo skirstymo į paketus šablonai}
Diskusijose, kaip reikėtų skirstyti programini kodą, paprastai akcentuojami du šablonai - pagal \textit{techninį sluoksnį}, todo: šaltiniai, nurodantys skirstymo būdus
kur kiekvienam funkcionalumui arba kompiuterinės sistemos sluoksniui yra sukuriamas paketas,
grupuojant skirtingų dalykinių sričių esybes, arba pagal \textit{dalykinės srities esybes}, kur vienos esybės kodas, dalykinės srities
esybės funkcionalumas skirtingose programiniuose sluoksniuose yra patalpintas viename pakete.
Tačiau šie du šablonai yra gan platūs ir galėtų būti išskaidyti į daugiau smulkesnių ir tiksliau aprašytų šablonų.
Taip pat, minėtuose šablonuose, būdai kaip ir kodėl skaidyti programinį kodą parinkti akecentuojant tai, kaip programinį
kodą supranta žmonės, dirbantys prie to kodo.
Nuspresti, kaip žmones supranta programinį kodą yra gan sudėtingas ir subjektyvus procesas, todėl aprašant šablonus, kodo skirstymui
geriau akcentuoti, kaip sugrupuoti paketai bendrauja tarpusavyje ir skirstyti juos pagal klasių naudojimo atvejus ir priklausomybes.
Taip kodo grupavimo metodai yra labiau artimi Martino aprašytiems principams.
Šablonus kaip grupuoti kodą, akcentuojant klasių naudojimo atvejus ir priklausomybes nagrinėja Martin Sandin savo
straipsnyje \textit{Four Strategies for Organizing Code}.
Šis straipsnis idomus tuo, kad autorius nesiplečia į du dažniausiai sutinkamus šablonus - grupuoti pagal techninį sluoksni arba dalykinės srities esybes,
o aprašo keturis grupavimo būdus arba šablonus, kurie, nors ir įkvėpti minėtų dviejų budų, yra gan unikalūs ir labiau techniškai apibrėžti.


\subsection{Pagal komponentą}
Organizavimas pagal komponentus sumažina sistemos sudėtingumą, pabrėždamas išorinę ir vidinę kodo vienetų darną.
Išorinė darna reiškia, kad paketas turi minimalią sąsają \angl{interface}, kuri atskleidžia tik konceptus (metodus arba duomenų tipus),
kurie yra glaudžiai susiję su komponento teikiama paslauga.
Vidinė darna reiškia, kad pakuotėje esantis kodas yra stipriai susijęs tarpusavyje ir susijęs su teikiama paslauga.

Kodas yra grupuojamas į mažus paketus, turinčius vieną, aiškiai apibrėžtą funkcionalumą ar tikslą, aprašant abstrakciją, kokie paketo elementai
yra pasiekiami iš išorės ir kaip jie naudojami.
Taip sukuriamas kodas, kuris yra lengviau suprantamas.
Tokią kodo grupavimo tvarką sunku palaikyti, tačiau jos rezultatas - kodas, kuris yra lengviau suprantamas, lengviau pagerinamas, lengviau testuojamas
ir, dėl aiškiai aprašytų sąsajų, lengviau pernaudojamas.

\begin{figure}[H]
    \centering
    \includegraphics[scale=0.2]{img/component_packaging}
    \caption{Sistemos sugrupuotos pagal komponentą pavyzdys}
    \label{img:component_packaging}
\end{figure}


\subsection{Pagal techninį sluoksnį}
Laikantis skirstymo pagal techninį sluoksnį, kiekvienam funkcionalumui arba kompiuterinės sistemos sluoksniui yra sukuriamas paketas,
kuris savyje grupuoja skirtingas dalykinės srities esybes. Pavyzdžiui, visos sąsajos darbui su duomenų baze guli viename pakete, sąsajos
su verslo logikos transformacijomis kitame, o duomenų vaizdavimo klientui logika trečiame pakete.
Šis paketų skirstymo metodas yra labai paplitęs, ypač tarp senesnių kompiuterinių sistemų. Jį paprasta įgyvendinti,
metode paprasta pavadinti paketus, aišku, į kuriuos paketus priskirti klases. Nors tokią kodo struktūrą lengva įgyvendinti ir palaikyti sistemoje,
ji turi nemažai trūkumų. Šis metodas nepalengvina sistemos plėtros valdymo - augant sistemai paprastai daugėja dalykinės srities esybių,
o ne funkcinių sluoksnių, todėl esamų paketų skaičius beveik nesikeičia, bet klasių kiekis pakete vis auga, tai daro neigiamą įtaką navigacijai
paketo viduje. Skirstant paketus pagal funkciją taip pat nėra gerinamas informacijos slėpimas (inkapsuliacija) - kiekvienas paketas savyje
talpina skirtingų dalykinės srities esybių kodą, tai sudaro salygas per klaidą užmegzti komunikaciją tarp komponentų, kurie neturėtų būti susiję.
Taip pat, skirstant tokiu būdu, sąsajos yra stipresnės tarp loginių komponentų, pasiskirsčiusių per sluoksnius, nei tarp vieno sluoksnio esybių.
Tokiu atveju pokyčių pristatymas pasidaro sudėtingas, nes reikalinga keisti ne vieną sluoksnį.
Tačiau šis metodas turi ne tik trūkumus - vienas iš jo privalumų - ganėtinai aiškiai nusakoma bendra sistemos architektūra, programiniai sluoksniai.
Deja, netvarkinga sistemos būsena eliminuoja ši privalumą.

\subsection{Pagal tipą}
Kodo organizavimas pagal tipą įprastai nėra griežtai apibrėžtas - klasės grupuojamos pagal vartotojo sumanytą tipą, neteikiant svarbos
klasių sąryšiams ar loginėms esybėms. Taip skirstant klases, į paketus galėtų būti grupuojamos esybių, išimčių ar serviso klasės. Šis skirstymo būdas
neteikia prioriteto nei skirstymui pagal techninius sluoksnius, nei pagal dalykinės srities esybes. Jį ganėtinai nesudėtinga įgyvendinti,
tačiau šis metodas yra netvarkingas bei turi trūkumų. Taip suskirstytas kodas sunkiai skaitomas, nes nėra aišku, pagal kokią tvarką
ieškoti konkrečios klasės ar kaip priskirti esamiems paketams netinkamas klases. Taip pat toks skirstymo būdas visiškai nepadeda spręsti
klasių sąsajų problemų, kadangi neapgalvotai išskirstyti sluoksniai gali būti glaudžiai susiję.

\subsection{Pagal funkciją}
Šis skirstymo būdas dalinai panašus į skirstymą pagal komponentą, tačiau yra mažiau griežtas ir
prioritetas teikiamas ne vidinei darnai ir glaudžiai grupuojamų
komponentų sąsajai, o išorinei darnai. Tokio skirstymo būdo paketuose dažniausiai grupuojamos tos pačios sąsajos implementacijos,
parenkamos siekiant pabrėžti išorinę darną ir sugrupuoti klases, teikiančias panašias funkcijas. Toks skirstymas patogus, pavyzdžiui,
techninių bibliotekų vartotojams, kadangi galima lengvai surasti bibliotekos teikiamas funkcijas. Tokią kodo grupavimo tvarką sunku palaikyti,
nes reikia gerai apgalvoti skirstymo strategiją, kad ji būtų prasminga ir patogi naudoti.


\section{Kodo skirstymo į paketus šablonų analizė}
Norint įvertinti, kuris šablonas duotai problemai spręsti yra tinkamiausias, reikalinga nagrinėti, kaip siūlomas sprendimas
sprendžia atitinkamą problemą ir kokias pasekmes gali turėti jo taikymas.

\subsubsection{Pagalbinių, daugkartinio naudojimo klasių skirstymo šablonų analizė}
Pagalbinių klasių skirstymui buvo išskirti du šablonai - kiekvienai esybei priskirtos pagalbinės klasės bei atskiras pagalbinių klasių paketas.
Svarbi problema, susijusi su pagalbinėmis klasėmis - su sistema dirbantys inžinieriai nežino apie jų egzistavimą, todėl jų nenaudoja,
tai veda prie didesnio kodo pasikartojimo arba kelių skirtingų to paties pagalbinio funkcionalumo įgyvendinimų.
Šią problemą sprendžia atskiras pagalbinių klasių paketas - naudojant tokį šabloną, programuotojas, susiduriantis su bendrine problema, kuri, tikėtina, jau yra išspręsta sistemoje, turėtų
aiškų procesą, kaip elgtis šioje situacijoje:
\begin{enumerate}
    \item Atsidaryti vieną paketą, skirtą bendrinio panaudojimo kodui
    \item Pakete surasti klasę, kurios pavadinimas būtų susijęs su jo problema
    \item Klasės funkcijų saraše surasti jam tinkamą funkciją.
    \item Jei reikalingas funkcionalumas nerastas, įgyvendinti jį pasirinktoje klasėje, padengti jį testais,
    bei aprašyti dokumentaciją, kaip funkcija turėtų būti naudojama.
    \item Iškviesti rastą arba sukurtą funkciją iš bendrinio panaudojimo kodo paketo savo funkcionalume
\end{enumerate}
Pakete reikėtų turėti atskiras klases kiekvienai bendrinei dalykinei sričiai, iš kurios pavadinimo programuotojas galėtų nuspresti,
kad jo ieškomas funkcionalumas bus būtent toje klasėje.
Tokiu atveju svarbu užtikrinti, kad iš klasių pavadinimo aišku, kokią smulkesnę dalykinės srities sritį padengia klasė, bei kad šios klasės
neturėtų priklausomybių nuo jas naudojančių klasių - kitu atveju sudaromos ciklinės priklausomybės.

Skirstant pagalbines klases po ja dengiamos esybės paketais, kyla kodo pasikartojimo rizika - nepatikrinus skirtingoms esybėms skirtų pagalbinių
klasių, sunku žinoti, ar toks funkcionalumas jau buvo įgyvendintas.

\subsubsection{Didelio klasių skaičius pakete skirstymo šablonų analizė}
Dideliam klasių pakete skaičiui spręsti buvo išskirti du šablonai - žemesnio lygio paketų sudarymas grupuojant pagal techninį sluoksnį ir
skirstymas pagal smulkų funkcionalumą.
Klasių skirstymas į paketus pagal techninį sluoksnį dažnu atveju yra mažiau pastangų reikalaujantis metodas - atskirti,
pavyzdžiui, kurios klasės priklauso \textit{service} sluoksniui yra paprasčiau, nei įvertinti, kurių klasių teikiamas funkcionalumas yra glaudžiau susijęs.
Tačiau šis būdas ne taip efektyviai prisideda prie lengviau suprantamos sistemos struktūros -
hibridinis skirstymo nesuteikia pilnos informacijos nei apie dalykinės srities esybes, nei apie technines sistemos dalis.
Taip pat, taikant šį metodą, gali kilti papildomų problemų - ne visas klases galima užtikrintai priskirti vienam sluoksniui.
Tokiu atveju, pavyzdžiui, esybės pagalbinėms klasėms, \textit{orchestrator} tipo klasėms reikėtų kurti papildomus paketus.

Skirstymą pagal smulkų funkcionalumą pagrindžia Robert C. Martin bendro sąryšio principas, kuris teigia, kad visos tarpusavyje susijusios klasės turėtų būti vienam pakete ir
akcentuoja siekiamybę turėti gan mažus paketus, turinčius aiškiai apibrėžtą funkcionalumą, priežastį egzistuoti, taip užtikrinant ir
glaudų tarpusavyje susijusių klasių saryšį.
Šis principas taip pat gali padėti užtikrinti aferentinių jungčių skaičių paketuose.
Didelis priklausomybių nuo specifinio paketo skaičius (arba aferentinės jungtys), reiškia, kad pokyčiai tame pakete turės įtaką kelioms klasėms.
Funkcionalumas kitiems paketams galėtų būti pasiekiamas per vieną minimalią sąsają \angl{interface},
kuri atskleidžia tik konceptus (metodus arba duomenų tipus), kurie yra glaudžiai susiję su komponento teikiama paslauga, bei
klase, grąžinančią minėtos sąsajos įgyvendinimą.
Paketas, turintis vieną funkciją, yra naudojamas tik tų paketų, kuriems reikia būtent tos funkcijos,
taip užtikrinant tik mažos sistemos dalies priklausomybę nuo vieno paketo.
Taip pat mažas paketo funkcionalumas reiškia, kad minėtas paketas skirtas funkcionalumui įgyvendinti naudos minimalų kitų sistemos esybių skaičių,
taip sumažinant ir eferentinių jungčių skaičių.

Žemiau esančiuose paveikslėliuose galima matyti, kaip išskaidant paketus, turinčius kelis funckionalumus, yra sumažinamas paketų
priklausomybių skaičius.
\begin{figure}[H]
    \centering
    \includegraphics[scale=0.15]{img/excesive_deps}
    \caption{Sistemos pavyzdys su kelias funkcijas atliekančiais paketais}
    \label{img:excesive_deps}
\end{figure}


\begin{figure}[H]
    \centering
    \includegraphics[scale=0.13]{img/good_deps}
    \caption{Sistemos pavyzdys su aiškią, vieną funkciją turinčiais paketais}
    \label{img:good_deps}
\end{figure}

Toks skirstymo būdas taip pat sprendžia ciklinių priklausomybių problemą - pavyzdžiui, įrankio, skirto spręsti ciklinių priklausomybių problemą,
kūrimo aprašas~\cite{CircularDependencies} teigia, kad ciklinių priklausomybių problemą galima spręsti laikantis trijų principų -
bendro panaudojimo, bendro keitimosi bei paleidimo ir pernaudojimo ekvivalentumo.
Šie principai yra išvesti iš bendro saryšio principo ir akcentuoja, kad kartu besikeičiančios klasės turėtų būti viename pakete.
Tokiu atveju ciklinių priklausomybių tikimybė sumažėja.

Šio šablono pritaikymas gali būti sudėtingesnis - norint išskaidyti didesnės apimties paketą į kelis mažesnius, gali būti
sudėtinga atskirti, koks grupavimas tinkamesnis.
Tačiau, nors šį principą sunkiau pritaikyti, jis gali padėti išspręsti ne tik didelio klasių skaičiaus pakete problemą,
bet ir prisidėti prie mažesnio priklausomybių skaičiaus, taip užtikrinant sistemos tvirtumą.


\subsubsection{Skirtingų sąsajų implementacijų skirstymo šablonų analizė}
Skirtingų sąsajų implementacijų skirstymui buvo išskirti du šablonai - sąsajų ir implementacijų grupavimas bei
įgyvendinimų atskyrimas.
Naudojant sąsajų ir implementacijų grupavimą, lengva rasti reikalingas implementacijas, tačiau šis būdas
turi vieną trūkumą - pakete su dideliu klasių kiekiu bei keliomis skirtingomis sąsajomis sunku suprasti, kuri klasė kurią sąsają įgyvendina.

Naudojant įgyvendinimų atskyrimą, galima labai greitai rasti sąsajas bei galimus jos įgyvendinimus beveik netyrinėjant sistemos struktūros bei klasių kodo.
Šis šablonas taip pat užtikrina, kad paketas turi vieną aiškų funkcionalumą, todėl laikosi bendro sąryšio principo.

\subsubsection{Esybių pokyčių ir versijavimo skirstymo šablonų analizė}
Esybių versijavimo skirstymui buvo išskirtas skirtingų versijų grupavimo į paketus šablonas.
Toks skirstymo būdas leidžia išvengti kodo duplikacijos bei perteklinio klasių skaičiaus paketuose, bei užtikrina aiškią skirtingų versijų
atskirtį. Toks būdas gali praplėsti skirstymą pagal dalykinės srities esybes - skirstomos versijos gali atspindėti reikalingus
palaikyti besikeičiančius verslo reikalavimus.

\subsection{Kodo skirstymo į paketus šablonų analizės išvados}
Išanalizavus skirtingus kylančias problemas sprendžiančius šablonus matoma, kad
pagalbinių klasių skirstymui tinkamesnis naudoti atskiras pagalbinių klasių paketas,
didelio klasių skaičiaus pakete skirstymui - skirstymas pagal smulkų funkcionalumą,
o skirtingų sąsajų implementacijoms - įgyvendinimų atskyrimą.

\section{Kompiuterinės sistemos vertinimas}

\subsection{Teisingai įgyvendinta kompiuterinė sistema}
Norint išsiaiškinti, kokią įtaką sistemos kokybei daro skirtingos paketų skirstymo metodologijos ir kaip ojektyviai tą įtaką pamatuoti, pirmiausia
reikėtų apsibrėžti kokiais požymiais pasižymi teisingai įgyvendinta kompiuterinė sistema.
Martin Kleppmann savo knygoje \textit{Designing Data-Intensive Applications: The Big Ideas Behind Reliable, Scalable, and Maintainable Systems} išskyria šiuos pagrindinius kriterijus:
\begin{itemize}
    \item Patikimumas, reiškiantis, kad net ir klaidų (įrangos, programinių ar žmogiškųjų) atveju,
    sistema veikia stabiliai ir patikimai, paslepiant tam tikras klaidas nuo vartotojo\cite{DataIntensiveApplications}.
    \item Prižiūrimumas, reiškiantis jog skirtingų abstrakcijų pagalba sumažintas sistemos kompleksiškumas.
    Dėl to nesunku keisti esamą sistemos funkcionalumą bei pritaikyti naujiems verslo naudojimo atvejams.
    Tai supaprastina darbą inžinierių ir operacijų komandoms dirbančioms su šia sistema, taip pat leidžia prie sistemos prisidėti naujiems žmonėms, o ne
    tik jos ekspertams.
    Tai ypač aktualu atviro kodo sistemoms\cite{DataIntensiveApplications}.
    \item Plečiamumas, reiškiantis jog sistema turi strategijas, kaip išlaikyti gerą našumą užklausų
    srautui didėjant ir sistemai augant, tai atliekant su pagrįstais kompiuteriniais resursais ir
    priežiūros kaina\cite{DataIntensiveApplications}.
\end{itemize}
Yra daug skirtingų elementų, sudarančių sistemą, kuri tenkintų aukščiau paminėtus kriterijus,
pavyzdžiui, pasirinktos technologijos, aukšto lygio architektūra, dokumentacija, sistemos testavimo
procesai, jų kiekis ir pan.
Vienas iš svarbių elementų, prisidedančių prie gerai įgyvendintos sistemos dizaino yra programinio kodo dizainas, jo skaitomumas, patikimumas.
Konvencijos, kaip vadinti kodo paketus, kokias klasės jiems priskirti ir kokios paketų hierarchijos laikytis sudaro svarbią programinio kodo dizaino dalį,
Todėl programinės įrangos kurimo metu, laikas skirtas rasti sistemai tinkamą paketų skirstymo šabloną ir to šablono laikymasis atsiperka, padarant
programinį kodą geriau suprantamu, taip prisidėdant prie bendro sistemos dizaino patikimumo ir lengvesnio palaikomumo.

Straipsnyje \textit{Investigating The Effect of Software Packaging on Modular Structure Stability}, jo autoriai akcentuoja, kad
gerai įgyvendintos, objektiškai orientuotos sistemos turėtų vystytis be didelių pakeitimu jų architektūroje.
To siekiama todėl, nes architektūriniai pakeitimai paveikią didelę sistemos dalį ir todėl
jų įgyvendinimo ir priežiūros kaštai yra ženkliai didesni\cite{ModularStability}.
Paketų struktūra, kuri užtikrina atsieta (\angl{decoupled}) komunikavima tarp paketų, enkapsuliuoja paketų vidinius elementus, neleidžiant pakeitimais
išplisti už paketų ribų yra pagrindas tvirtai sistemos architektūrai, gebančiai efektyviai plėstis, ženkliai nesikeičiant ir sutaupant programos priežiūros kaštus.

\subsection{Kodo skirstymo paketais metodų vertinimas}
Ankstesniame skyriuje buvo nagrinėjama gerai įgyvendintos paketų struktūros įtaka geram kompiuterinės sistemos dizainui,
tačiau lieka neatsakytas klausimas - kaip įvertinti metodą kodui į paketus grupuoti, kaip objektyiviai užtikrinti
jog šis sprendimas yra butent toks kokio reikia ir kokia būtent jo įtaka kompiuterinei sistemai?
Tvarkingas, aiškiai suprantamas kodas yra subjektyvi tema, priklausanti nuo komandos,
naudojamos programavimo kalbos ar programinių įrankių bei programinės sistemos dalykinės srities.
Kodo grupavimo į paketus metodai, taip, kaip ir bendros tvarkingo kodo praktikos,
gali būti labai subjektyvūs ir patogus tik metodą formavusiam asmeniui.
Tam, kad pagrįstai įvertinti skirtingus kodo skirstymo šablonus, pasiekiant kuo objektyvesnį,
plačiau priimtiną rezultatą, reikėtų aprašyti kriterijus, ko tikimasi iš paketų struktūros.

Robert C. Martin savo knygoje \textit{Agile Software Development, Principles, Patterns, and Practices} aprašo
šesis principus padedančius teisingai grupuoti klases į paketus.
Rodiklis, kiek kiekvienas paketas sistemoje laikosi nurodytų principų, tam tikrame paketų skirstymo šablone, gali
būti kriterijus, įvertinti to šablono kokybei ir įtakai bendram sistemos dizainui.

Pirmi trys principai yra apie paketo elementų sąryšį, jie padeda paskirstyti klases paketams\cite{AgileSoftwareDevelopment}.
Kiti trys principai reglamentuoja paketų sąjungą, jie padeda nustatyti, kaip paketai turėtų būti tarpusavyje susiję.

Paketų stabilumo metrikos:
\begin{itemize}
    \item \textit{Klasių skaičius} - klasių skaičiaus metrika paketui nurodo, skaičių klasių (konkrečių ir abstrakčių) pakete.
    Ši metrika matuoja paketo dydį.
    \item \textit{Aferentinės jungtys \angl{Afferent Couplings}} - aferentinių jungčių metrika nurodo
    skaičių kitų paketų, kurie priklauso nuo klasių esančių pasirinktame pakete.
    Ši metrika matuoja ateinančias priklausomybes.
    \item \textit{Eferentinės jungtys \angl{Efferent Couplings}} - eferentinių jungčių metrika nurodo skaičių kitų paketų,
    nuo kuriu priklauso klasės pasirintame pakete.
    Ši metrika matuoja išeinančias priklausomybės.
    \item \textit{Nestabilums} - nestabilumo metrika nurodo santyki tarp eferentinių jungčių ir
    visų jungčių (Aferentinės + Eferentinės) pakete.
    Ši metrika matuoja paketo atsparumui pokyčiams.
    Reikšmės rėžiai - nuo nulio iki vieno, kur vienas nurodo visiškai stabilų paketą, o vienetas visiškai nestabilų.
    \item \textit{Atstumas} - Atstumo metrika apibrėžiamas kaip statmenas pakuotės atstumas nuo idealizuotos linijos (A + I = 1),
    kur A yra abstrakčių klasių procentas nuo bendro paketo klasių skaičiaus.
    Ši metrika yra paketo abstraktumo ir stabilumo pusiausvyros rodiklis.
    Paketas tiesiai pagrindinėje sekoje yra optimaliai subalansuotas, atsižvelgiant į jos abstraktumą ir stabilumą.
    Šios metrikos diapazonas yra nuo nulio iki vieneto, o nulis nurodo paketą, kuris sutampa su pagrindine seka,
    o vienas – paketą, kuris yra kuo toliau nuo pagrindinės sekos. todo: pataisyti
    \item \textit{Žiedinės priklausomybės} - žiedinių priklausomybių metrika skaičių atveju, kur pasirinkto paketo išeinančios priklausomybes taip pat
    yra paketo ateinančios priklausomybes (tiesiogiai arba netiesiogiai)
\end{itemize}
\section{Įrankiai šablonų analizei ir įvertinimui}
Kompiuterinės sistemos, kurioms yra aktualu klasių ir paketų skirstymo metodai, paprastai yra labai didelės.
Pilnai perprasti tokias sistemas, nustatyti jų įgyvendinimo kokybę, naudojamus šablonus kodui skirstyti,
apskaičiuoti paketų kokybes metrikas yra sudėtingas procesas.
Daryti tai rankiniu budu užtrunka daug laiko bei yra paliekama daug vietos potencialioms klaidoms,
todėl yra būtina šį procesą optimizuoti, skaitmenizuoti analizės procedūras pasitelkiant
aiškiai apibrėžtų ir programiškai efektyvių programinių įrankių pagalbą.

\subsection{Reikalavimai įrankiams}
Įrankių, leidžiančių paprasčiau atlikti sistemų analizę, atsakomybes galima suskirstyti į dvi grupes:
\begin{itemize}
    \item Bendrinė sistemos analizė - įrankis ar įrankiai padeda atlikti bendrinę sistemos analize.
    Šios atsakomybių grupės įrankių išvestis nėra objektyvūs, tiksliai apibrėžtų formulių rezultatai, o papildoma, aiškiai
    atvaizduota, meta informacija apie sistema - paketų struktūrą, jų priklausomybes, figuruojančių paketų bei klasių vardai.
    Ši papildoma informacija nėra aiškūs teiginiai, o tik pagalba analizę atliekančiams asmeniui, leidžianti priimti ižvalgas apie sistemą,
    kaip pavyzdžiui, kokiam paketų skirstymo šablonui yra atimiausia sistemos strukūra, arba kaip lengvai sistema yra suprantama.
    \item Paketų kokybės metrikų skaičiavimas - įrankis ar įrankiai turi padeda paskaičiuoti aprašytas paketų kokybes metrikas.
    Šių įrankių išvestis - tikslūs, formulėmis pagrįstų skaičiavimų rezultatai apie paketų kokybę, kuriuos galima lyginti tarpusavyje.
\end{itemize}

todo: akcentuoti kad visas darbas vyksta su java sistemomis
\subsection{Reikalavimai įrankiui bendrinės sistemos analizei}
Įrankis bendrinei sistemos analizei atlikti, turėtų suteikti galimibė naudotojui nurodyti kelią iki \texit{Java} programavimo kalba parašytos sistemos arba posistemės ir joje
atlikti jos turinio analizę bei naudotojui pateikti naudingas išvadas, sudarytas iš:
\begin{itemize}
    \item Klasių ir paketų skaičiaus
    \item Vidutinį klasių pakete skaičiaus
    \item Paketų ir klasių medį, identifikuojanti abstrakčias klases ar sąsajas
    \item Paketų priklausomybių grafiką
\end{itemize}
Gautą rezultatą išvesti suprantamu formatu, leidžiant vartotojui priimti išvadas apie sistemos, arba tam tikros
posistemės struktūrą, naudojamus įrankius bei kokybę.

\subsection{Reikalavimai įrankiui paketo kokybėi skaičiuoti}
Įrankis paketo kokybei skaičiuoti, turėtų suteikti galimibė naudotojui nurodyti kelią iki \texit{Java} programavimo kalba parašytos sistemos arba posistemės ir joje
apskaičiuoti kiekvieno paketo kokybės metrikas:
\begin{itemize}
    \item Klasių skaičius
    \item Aferentinių jungčių skaičius
    \item Eferentinių jungčių skaičius
    \item Nestabilumo santykis
    \item Abstrakcijos santykis
    \item Atstumo nuo pagrindinės sekos santykis
    \item Žiedinių priklausomybių skaičius
\end{itemize}
Gautą rezultatą išvesti vartotojui suprantamu formatu, kuriame matytųsi individualių paketų metrikos, bei šių metrikų vidurkis sistemoje (arba posistemėje).
Išvedimo formatas turėtų būti toks, jog skirtingų analizių rezultatai būtų lengvai palyginami su kitais.

\textbf{Abiejose vienas iš palaikomų išvesties formatų turėtų būti \testit{latex}, taip suteikiant galimybę analizes rezultatus pateikti tolesniame šio dokūmento tūrinyje.}

\subsection{Įrankių įgyvendinimas}
Nors beveik visiems reikalavimuose minimiems funkcionalumas galima rasti jau sukurti įrankių, greit ir efektyviai pritaikyti visus juos skirtingom sistemoms
(arba posistemėms) nėra patogų - kiekvieną įrankį reikėtų vykdyti atskirai, su skirtingais vykdymo procesas ir argumentais, taip išvestų rezultatų formatai yra skirtingi.
Todėl, norint palengvinti ši procesą - suvienodinti procesų vykdymą, bei gautus rezultus, visi įrankiai reikalingi analizei, įgyvendinti kaip viena programinė sistema, kuri
apdoroja failus nurodytoje sistemos direktorijoje, nuskaito \textit{java} failų tūrinį ir sukonstruoja informaciją apie sistemos paketųs bei klases.
Surinkta informacija tada naudojama įgyvendinti kiekvienam aprašytam įrankio funkcionalumui, ten kur galima naudojant jau parašytus įrankius, taip programiškai supaprastinant
skirtingų įrankių vykdymą.

%\section{Kodo skirstymo metodai realiose sistemose}
\subsection{Sistemų pasirinkimas}

\subsection{Sistemų analizės procesas}

\section{Esamų sistemų pertvarkymas}
Apibrėžus šablonus paketų skirstymui galima pradėti vertinti jų efektyvumą, juos pritaikant esamoms sistemoms.
Šio skyriaus tikslas - pasirinkti ir išnagrinėti kelias sistemas, pasitelkus paketų kokybes matus, bei bendrą sistemos struktūros analizę,
taip identifikuojant programiniam kodui būdingas problemas, rastas problemas išspręsti pritaikant aprašytus šablonus.
Atlikus pertvarkymus sistemose, dar kartą paskaičiuoti paketų kokybės matus,
taip gaunant įrodymus, ar gauti šablonai yra efektyvūs ir iš tiesų sprendžia
jiems priskirtas problemas.


\subsection{Sistemų pasirinkimas}
Sistemų pertvarkymui pasirinktos atviro kodo sistemos, kurių kodas yra viešai prietinamas \textit{github} platformoje.
Pasirinktos sistemos yra vidutinio dydžio, todėl nėra labai sudėtinga jas suprasti ir pertvarkyti, bet taip pat jos nėra
tokios paprastos, kad neturėtų sistemos dizaino problemų.
Pasirinktos sistemos yra skirtingo tipo projektai, taip užtikrinant didesnę problemų ivairovę ir objektyvesnius įvertinimus.
Per visą pasirinktų sistemų imtį yra sutinkamos visos aprašytos problemos, taip įvertinant visus aprašytus šablonus.

\subsection{\textit{Leaf} sistema}
\textbf{Leaf\footnote{\url{https://github.com/Meituan-Dianping/Leaf/tree/master}}} sistema, per \textit{http}
protokolą teikiantį aplikacijų programavimo sąsają unikalaus identifikatoriaus generavimui.
Ši sistema užtikrina identifikatoriaus unikalų pasiskirstymą, tarp skirtingų, paskirstytų sistemų \angl{distributed systems},
servisais orientuotoje \angl{service-oriented} architektūroje.
Tai techninės programinės įrangos tipas, kuris naudojamas kitų taikomosios programinės įrangos sistemų.
Sistema Leaf susideda iš dviejų modulių \textit{server} ir \textit{core}, šiame darbe dėmesys bus skirtas tik \textit{core} moduliui,
kadangi \textit{server} modulis yra labai mažas
ir paketų struktūra jame neatlieka esminio vaidmens.
Prieš visus pakeitimus \textit{Leaf} sistemos, \textit{core} modulio paketų struktūra atrodo taip:
\begin{figure}[H]
    \centering
    \includegraphics[scale=0.5]{img/leaf_packages_orig}
    \caption{\textit{Leaf} sistemos \textit{core} modulio struktūra}
    \label{img:leaf_packages_orig}
\end{figure}

\dirtree{%
.1 {/}.
.2 {common}.
.2 {segment}.
.3 {dao}.
.4 {impl}.
.3 {model}.
.2 {snowflake}.
.3 {exception}.
}

\begin{center}
    \begin{tabular}{|c|c|c|c|c|c|c|}
        \hline
        Paketo vardas & \textit{N} & \textit{A} & \textit{E} & \textit{S} & \textit{A} & \textit{D} \\ [0.5ex]
        \hline\hline
        leaf.segment.dao.impl & 1 & 0 & 2 & 1.0 & 0.0 & 0.0 \\
        \hline
        leaf.segment & 1 & 0 & 4 & 1.0 & 0.0 & 0.0 \\
        \hline
        leaf.snowflake.exception & 3 & 1 & 0 & 0.0 & 0.0 & 1.0 \\
        \hline
        leaf.segment.model & 3 & 3 & 0 & 0.0 & 0.0 & 1.0 \\
        \hline
        leaf.segment.dao & 2 & 2 & 1 & 0.333 & 1.0 & 0.333 \\
        \hline
        leaf & 1 & 3 & 1 & 0.25 & 1.0 & 0.25 \\
        \hline
        leaf.common & 6 & 3 & 1 & 0.25 & 0.0 & 0.75 \\
        \hline
        leaf.snowflake & 2 & 0 & 3 & 1.0 & 0.0 & 0.0 \\
        \hline
    \end{tabular}
    \begin{tabular}{|c|c|c|c|c|c|}
        \hline
        $\bar{N}$ & $\bar{A}$ & $\bar{E}$ & $\bar{S}$ & $\bar{A}$ & $\bar{D}$ \\ [0.5ex]
        \hline\hline
        2 & 2 & 2 & 0.479 & 0.25 & 0.417 \\
        \hline
    \end{tabular}
\end{center}


\sectionnonum{Rezultatai}
\begin{enumerate}
    \item Išanalizavus atviro kodo kompiuterines sistemas, išskirti kodo skirstymo šablonai projektavimo metu kylančioms problemoms spręsti.
    \item Sukurta programa, išvedanti bendrą informaciją apie sistemą bei matų įverčius.
    \item Pagal šablonus pertvarkytos sistemos, šablonai įvertinti pagal pasirinktus matus.
    \item Pateiktos rekomendacijos šablonų taikymui
\end{enumerate}

\sectionnonum{Išvados}
\begin{enumerate}[labelindent=0pt]
    \item Skirstant klases į paketus pagal dalykinę sritį gali iškilti situacijų, kurioms šis metodas nėra pakankamas.
    \item Problemų, kurias sudėtinga išspręsti skirstant kodą į paketus pagal dalykinę sritį, sąrašas nėra baigtinis, tačiau
    aprašytos problemos pasikartoja ne vienoje sistemoje, ir yra sprendžiamos skirtingais būdais.
    \item Rekomenduojami naudoti šablonai turi teigiamos įtakos vertinant pagal matus, siejamus su lengvesniu sistemos suprantamumu bei palaikomumu.
\end{enumerate}

\printbibliography[heading=bibintoc]  % Šaltinių sąraše nurodoma panaudota
% literatūra, kitokie šaltiniai. Abėcėlės tvarka išdėstomi darbe panaudotų
% (cituotų, perfrazuotų ar bent paminėtų) mokslo leidinių, kitokių publikacijų
% bibliografiniai aprašai. Šaltinių sąrašas spausdinamas iš naujo puslapio.
% Aprašai pateikiami netransliteruoti. Šaltinių sąraše negali būti tokių
% šaltinių, kurie nebuvo paminėti tekste (LaTeX tai sutvarko automatiškai).
% Šaltinių sąraše rekomenduojame necituoti savo kursinio darbo, nes tai nėra
% oficialus literatūros šaltinis. Jei tokių nuorodų reikia, pateikti jas tekste.

% Priedai
% Prieduose gali būti pateikiama pagalbinė, ypač darbo autoriaus savarankiškai
% parengta, medžiaga. Savarankiški priedai gali būti pateikiami ir
% kompaktiniame diske. Priedai taip pat numeruojami ir vadinami. Darbo tekstas
% su priedais susiejamas nuorodomis.
%\appendix{Neuroninio tinklo struktūra}
%
%\begin{figure}[H]
%    \centering
%    \includegraphics[scale=0.5]{img/MLP}
%    \caption{Paveikslėlio pavyzdys}
%    \label{img:mlp}
%\end{figure}


%\appendix{Eksperimentinio palyginimo rezultatai}
%
%% tablesgenerator.com - converts calculators (e.g. excel) tables to LaTeX
%\begin{table}[H]\footnotesize
%  \centering
%  \caption{Lentelės pavyzdys}
%  {\begin{tabular}{|l|c|c|} \hline
%    Algoritmas & $\bar{x}$ & $\sigma^{2}$ \\
%    \hline
%    Algoritmas A  & 1.6335    & 0.5584       \\
%    Algoritmas B  & 1.7395    & 0.5647       \\
%    \hline
%  \end{tabular}}
%  \label{tab:table example}
%\end{table}

\appendix{Nuorodos į nagrinėtas atviro kodo sistemas}
\begin{itemize}
    \item https://github.com/brettwooldridge/HikariCP
    \item https://github.com/h2database/h2database
    \item https://github.com/Meituan-Dianping/Leaf
    \item https://github.com/OpenHFT/Chronicle-Map
    \item https://github.com/ReactiveX/RxJava
    \item https://github.com/yasserg/crawler4j/tree/master
    \item https://github.com/google/guice
    \item https://github.com/Azure/azure-sdk-for-java/tree/main/sdk/storage/azure-storage-blob-cryptography
    \item https://github.com/elastic/elasticsearch
    \item https://github.com/apache/incubator-seata
    \item https://github.com/mongodb/mongo
    \item https://github.com/nocodb/nocodb
    \item https://github.com/mariazevedo88/travels-java-api
    \item https://github.com/AdoptOpenJDK/openjdk-jdk11
    \item https://github.com/keystonejs/keystone
    \item https://github.com/wix-incubator/source-wizard
    \item https://github.com/dbeaver/dbeaver
    \item https://github.com/hackjutsu/Fire\_Sticker
    \item https://github.com/federicoiosue/Omni-Notes
\end{itemize}
\end{document}
